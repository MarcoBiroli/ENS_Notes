\documentclass[10pt,a4paper]{book}
\usepackage[utf8]{inputenc}
\usepackage[english]{babel}
\usepackage{amsmath}
\usepackage{amsfonts}
\usepackage{physics}
\usepackage{amssymb}
\usepackage{graphicx}
\usepackage[left=2cm,right=2cm,top=2cm,bottom=2cm]{geometry}
\title{Hydrodynamic lecture notes}
\author{Marco Biroli \& Alessandro Pacco}
\begin{document}
\maketitle

\chapter{Fundamentals}
\section{Fluid particles and mechanics of continuous media.}

The continuous media approximation consists of placing oneself at a large enough scale so that the system can be considered continuous. To do so we must consider an elementary volume, this elementary volume is called the fluid particle. It has a mesoscopic size, so it contains enough particles for its statistics to be at equilibrium. However it is small enough for its properties to be defined locally. This elementary volume can undergo geometric transformations: rotation, translation, deformation etc. It is also submitted to forces. Forces can be of two types, volumic forces like gravity or electromagnetic forces and surface forces like pressure or viscosity. When we study a system we need to consider the forces acting upon it, when we do so we always consider the forces acting upon a fluid particle. Once all this is defined there are usually 2 types of viewpoints from which we can study the problem. The Eulerian standpoint is to, as an outside observer, define the properties as a map i.e. $\vec{v}(\vec{r}, t)$, this is almost always what is done. The Lagrangian standpoint is to define properties for a given fluid particle and follow it i.e. $x^{(p)} (t, x_0^{(p)})$. The fundamental difference of these two viewpoints is when we try to find the variation of a certain value $G$. In the Eulerian standpoint it is simply the derivative of the value in space. For a certain value that depends on the position of a fluid particle: $G(\vec{r}(t), t)$ we have:
\[
(1 \text{D for simplicity}) \quad \dd G = \pdv{G}{t} \dd t + \pdv{G}{x} v \dd t \Leftrightarrow \underbrace{\dv{G}{t}}_{\text{Lagrangian representation}} = \underbrace{\pdv{G}{t} + v \pdv{G}{x}}_\text{Eulerian representation}
\]

\section{Mass conservation.}
The mass of a fluid particle is considered to be constant hence:
\[
\dv{M}{t} = 0 \Rightarrow \dv{}{t}\int_{V} \rho \dd \tau = 0 \Rightarrow \int_V \pdv{\rho}{t} \dd \tau + \oint_S \rho  u \dd S = 0 \Rightarrow \pdv{\rho}{t} + \grad (\rho \vec{u}) = 0
\]



\end{document}