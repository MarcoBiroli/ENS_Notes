\documentclass[10pt,a4paper]{article}
\usepackage[utf8]{inputenc}
\usepackage[english]{babel}
\usepackage{amsmath}
\usepackage{amsfonts}
\usepackage{amssymb}
\usepackage{graphicx}
\usepackage{physics}
\usepackage[left=2cm,right=2cm,top=2cm,bottom=2cm]{geometry}
\author{Marco Biroli}
\title{Report - Condensed Matter}
\begin{document}
\maketitle
\section{Introduction.}
First and foremost, what is condensed matter? Condensed matter is on the interface in between the microscopic and macroscopic scale. Common examples are: ... . A good way to define condensed matter is the physics that studies the spacial and time scales of the atomic to the mesoscopic scale. The goal is to study the physical properties of organized forms of matter, relate microscopic behavior to the macroscopic properties (and vice-versa) and finally to lean about complex collective behavior. The complexity of the problems usually studied is humongeous since (even considering atoms like point like masses) there is a incredible amount of configurations that can be formed. Then if we want to introduce the quantim mechanical aspect we get a hamiltonian of the form:
\[
\hat{H} = -\sum_i \frac{\hbar^2}{2m} \grad_i^2 + \frac{1}{2}\sum_{i \neq j} \frac{e^2}{|r_i - r_j|} + \sum_{i, I} \frac{Z_I e^2}{|r_i - R_I|} - \sum_{I} \frac{\hbar^2}{2M_I} \grad_I + \frac{1}{2}\sum_{I \neq J} \frac{Z_J Z_J}{|R_I - R_J|}
\]
"More is different" 
Physical review letters are composed of almost 50\% by Condensed Matter. The main pillars of condensed matter today: The Fermi Liquid theory, The Landau symmetry-breaking theory of phase transitions. The main challenges today are:
\begin{itemize}
\item High temperature super conductors
\item Colossal magnetoresistive manganites
\item FQH states of 2DEG
\item Graphene
\item Topologial order
\item Spin liquids
\item Majorana fermions
\item Lutting liquids.
\end{itemize}
Condensed matter is also on the frontier of technological innovation today.

\section{.}


\end{document}