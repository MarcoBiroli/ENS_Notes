\documentclass[10pt,a4paper]{book}
\usepackage[utf8]{inputenc}
\usepackage[english]{babel}
\usepackage{amsmath}
\usepackage{amsfonts}
\usepackage{amssymb}
\usepackage{stmaryrd}
\usepackage[left=2cm,right=2cm,top=2cm,bottom=2cm]{geometry}
\author{Marco Biroli}
\title{Dummit and Foote Exercises.}
\newcommand{\lcm}{\text{lcm}}
\begin{document}
\maketitle
\chapter{Preliminaries.}
\section{}
\begin{itemize}
\item We know that:
\[
20 = 2^2 \cdot 5 \quad \text{ and } \quad 13 \text{ is prime.}
\]
So $(20, 13) = 1$ and $\lcm(20,13) = 260$ and $2\cdot 20 - 3 \cdot 13 = 1 = (20, 13)$.
\item Similarly we have:
\[
69 = 3 \cdot 23 \quad \text{ and } \quad 372 = 2^2 \cdot 3 \cdot 31 \quad \text{ so } (69, 372) = 3 \quad \text{ and } \lcm(69, 372) = 2^2 \cdot 3 \cdot 23 \cdot 31 = 8556
\]
We also have:
\[
372 = 69 \cdot 5 + 27, \quad 69 = 27 \cdot 2 + 15, \quad 27 = 15 \cdot 1 + 12, \quad 15 = 12\cdot 1 + 3, \quad 12 = 4 \cdot 3
\]
So back-feeding this we have:
\[
3 = 15 - 12 = 2\cdot 15 - 27 = 2 \cdot 69 - 5 \cdot 27 = 27 \cdot 69 - 5 \cdot 372
\]
\item \ldots
\end{itemize}
\section{}
Suppose that $k | a$ and $k | b$ then $a = c \cdot k$ and $b = e \cdot k$. So $as + bt = c k s + e k t = k (cs + et)$ therefore $k | (as + bt)$.
\section{}
Suppose $n$ is composite, so $n$ is not prime and can be written $n = \sum_{i = 1}^k p_i^{\alpha_i}$. Then take $a = \sum_{i = 1}^{\lfloor \frac{k}{2} \rfloor} p_i^{\alpha_i}$ and $b = \sum_{\lceil \frac{k}{2} \rceil}^k p_i^{\alpha_i}$. Then it is clear that $n | ab$ but $n \not{|} a$ and $n \not{|} b$.
\section{}
Let $a,b, N$ be fixed integers with $a, b$ non-zero, set $d= (a,b)$ and suppose $x_0, y_0$ are particular solutions to $ax + by = N$. Then notice that taking $x = x_0 + \frac{b}{d} t$ and $y = y_0 - \frac{a}{d}t$ gives:
\[
ax + by = ax_0 + by_0 + \frac{abt - abt}{d} = N
\]
\section{}
\begin{center}
\begin{tabular}{| c | c | c | c | c | c | c | c | c | c |}
\hline
$\varphi(1)$ & $\varphi(2)$ & $\varphi(3)$ & $\varphi(4)$ & $\varphi(5)$ & $\varphi(6)$ & $\varphi(7)$ & $\varphi(8)$ & $\varphi(9)$ & $\varphi(10)$\\
\hline
1 & 1 & 2 & 2 & 4 & 2 & 6 & 4 & 6 & 4\\
\hline 
\end{tabular}
\end{center}

\section{}
We prove the well-ordering of $\mathbb{Z}^+$ by induction on the cardinality of the set $A$. The base case is trivial, now take a subset $A$ of $\mathbb{Z}^+$ of cardinality $n$. Then take any element $x \in A$. If $\forall m \in A,\,x < m$ we are done. Otherwise it means that $\exists y \in A,\, y < x$. Then take $B = A \setminus \{x\}$ by induction there is a minimal element $z \in B$ and from definition $z < y < x$ so $\forall m \in A,\, z < m$. This concludes the proof.

\section{}
Take $p$ a prime. Suppose there exist $a, b$ integers such that $a^2 = p b^2$, then $p | a^2$, since $p$ is prime this means that $p | a$. So $k^2 p^2 = p b^2$ therefore $ k^2 p = b^2$. By the same reasoning we get that $p | b$ so we need $k^2 p = m^2 p^2$ which gives that $p | k$, repeating this argument recursively we arrive at a contradiction.

\section{}
Let $p$ a prime. Take $n \in \mathbb{Z}^+$ then $n$ can be written as: $\sum_{i = 1}^k p_i^{\alpha_i}$. Now suppose that $p^\beta | n!$, by the properties of $p$ this means that $p^\beta$ must divide $n$ or $(n-1)!$. Repeating this recursively gives that $p^\beta$ must divide at least one $(n - i)$ with $i \in \llbracket 0, n-1 \rrbracket$.

\chapter{Introduction to Groups.}
\section{Intro}
\subsection{}
\begin{itemize}
\item $ a \star b = a - b$ is not associative since $a \star (b \star c) = a - (b - c) = a- b + c \neq a - b - c = (a - b) - c = (a \star b) \star c$.
\item $a \star b = a + b + ab$ is associative since:
\begin{align*}
a \star (b \star c) &= a\star( b + c + bc) = a + b + c + bc + ab + ac + abc\\
(a \star b) \star c &= (a + b + ab) \star c = a + b + ab + c + ac + bc + abc
\end{align*}
\end{itemize}

\section{Dihedral Groups.}
\subsection{}
Take $x \in D_{2n}$ that is not a power of $r$ then $x$ can be written as $r^k s$. So $r x = r^{k+1} s = r^k (rs) = r^k (sr^{-1}) = x r^{-1}$.
\subsection{}
Take $x \in D_{2n}$ that is not a power of $r$ then $x^2 = r^k s r^k s = r^k s s r^{-k} = r^k r^{-k} = 1.$
\subsection{}
Let $x, y$ be any elements of order 2 in any group $G$ not that this is equivalent to $x = x^{-1}$ and $y = y^{-1}$. Suppose that $t = xy$, then $tx = xyx$ and $xt^{-1} = xy^{-1}x^{-1} = xyx$. So $tx = xt^{-1}$.
\subsection{}


\end{document}