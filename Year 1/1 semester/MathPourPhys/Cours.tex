\documentclass[10pt,a4paper]{article}
\usepackage[utf8]{inputenc}
\usepackage[french]{babel}
\usepackage{amsmath}
\usepackage{amsfonts}
\usepackage{amssymb}
\usepackage{amsthm}
\usepackage{mathtools}
\usepackage[left=2cm,right=2cm,top=2cm,bottom=2cm]{geometry}
\author{Marco Biroli}
\title{Mathematiques pour physiciens.}
 
\theoremstyle{definition}
\newtheorem{definition}{Definition}[section]
 
\theoremstyle{theorem}
\newtheorem{prop}{Proposition}[section]
\newtheorem{theorem}{Theorem}[section]
\newtheorem{corollary}{Corollary}[theorem]
\newtheorem{lemma}[theorem]{Lemma}

\theoremstyle{remark}
\newtheorem*{remark}{Remark}
\newtheorem*{notation}{Notation}

\newcommand{\K}{\mathbb{K}}
\newcommand{\R}{\mathbb{R}}
\newcommand{\Tr}{\text{Tr}}
\newcommand{\trans}[1]{\prescript{t}{}{#1}}
\newcommand{\card}[1]{\text{Card }\left( #1 \right)}
\newcommand{\Id}{\text{Id}}
\newcommand{\Com}{\text{Com}}

\begin{document}
\maketitle
\section{Analyse Complexe.}
\subsection{Rappels.}
\definition Un ensemble $E$ est connexe ssi il ne peut être écrit comme la reunion de deux ouverts disjoints.
\prop L'image d'un connexe par une application continue est connexe.
\prop Les seuls ouverts-fermés de E sont $E$ et $\emptyset$.
\theorem $f$ analytique sur $U$ ouvert connexe. Si $f$ n'est pas la fonction nulle, si $z_0$ est un zéro de $f$ alors
$$
\exists \epsilon > 0 | z_0 \text{ soit le seul zéro de } f_{|D(z_0, \epsilon}
$$
\theorem Si la restriction de 2 fonctions analytiques sur $U$ ouvert coincide sur $U \subset D$ connexe alors ces fonctions sont égales sur $D$.
\theorem Si $U$ ouvert connexe, si $f$ atteint une maximum local sur $U$ alors $f$ est constante.

\subsection{Théorème de Cauchy.}
\subsubsection{Chemins et lacets.}
\definition Soit $\gamma : I  = [a, b] \rightarrow \mathbb{C}$, $\mathcal{C}^1$ par morceaux, alors $\gamma'$ est le vecteur tangent à la courbe. On appele $\gamma(a)$ l'origine, $\gamma(b)$ l'extremité et $\gamma$ le chemin.
\definition Un lacet est un chemin tq l'éxtrémité et l'origine sont confondues.
\definition Le chemin opposé à $\gamma$ est:
$$
\gamma^0 : t \longmapsto \gamma(a+b-t)
$$
\definition On dit que $\gamma_1$ et $\gamma_2$ sont equivalents si il existe une bijection $\varphi$ croissante $\mathcal{C}^1$, $\varphi^{-1}$ soit aussi $\mathcal{C}^1$:
\[
\gamma_2(t) = \gamma_1(\varphi(t))
\]
\subsection{Intégration le long d'un chemin.}
\definition Si $\gamma : I \Rightarrow \mathbb{C}$ est un chemin, $f$ continue sur $\gamma(I)$ alors $t \Rightarrow f(\gamma(t))\gamma'(t)$ et on définit:
$$
\int_\gamma f(z) dz = \gamma_a^b f(\gamma(t))\gamma'(t) dt
$$
\begin{remark} Cette définition coincide avec la circulation du champ vectoriel $(\Re f, \Im f)$ le long de $\gamma(I)$. De plus, on constate $\int_\gamma f = - \int_{\gamma^0} f.$ Aussi si $\gamma$ est un lacet alors l'integrale ne depend pas du point de départ choisi.
\end{remark}

\definition On aussi un règle de la chaine: si $\Gamma : t \rightarrow u(\gamma(t))$, $u$ analytique alors $\int_\gamma f(u(z))u'(z) dz = \int_\Gamma f(z) dz$.

\subsection{Primitives.}
\theorem Pour qu'une fonction analytique sur un domaine $D$ ouvert admette une primitive sur $D$ il faut et il suffi que pour tout lacet $\int_\gamma f = 0$.
\prop Lorsqu'il en est ainsi une primitive $F$ de $f$ s'écrit $F(z) = C + \int_\gamma f(w) d w$ ou $\gamma$ est un chemin d'origine arbitraire d'extrimite $z$. 
\begin{remark}
L'exemple récurrent est $f : z \mapsto \frac{1}{z}$. Alors soit $\gamma : [0, 2\pi] \to U, t\mapsto e^{it}$. Alors:
$$
\int_\gamma f(z) dz = \int_0^\pi f(e^{it}) i e^{it} dt = i \int_0^\pi dt = 2 \pi i 
$$
\end{remark}
\subsection{Homotopie}
\definition Une homotopie entre deux chemins $\gamma_0 : [a,b] \rightarrow \mathbb{C}$ et $\gamma_1 : [c, d] \rightarrow \mathbb{C}$ est une application continue:
$$
\varphi : I \times J \rightarrow D
$$
telle que $\varphi(t,c) = \gamma_0(t)$ pour tout $t$ dans $I$ et $\varphi(b, t) = \gamma_1(t)$ pour tout $t$ dans $J$.
\begin{remark}
On note que si on defini $\gamma_s (t) = \varphi(s,t)$ peut être vu comme une famille de lacets qui morphes du chemin $\gamma_0$ vers $\gamma_1$.
\end{remark}
\definition $D$ connexe est simplement connexe ssi tout lacet est homotope à un point.

\subsection{Théorème de Cauchy.}
\theorem Soit $D$ ouvert convexe, $f$ analytique sur $D$ alors pour tout lacet $\gamma \subset D$ on a:
$$
\int_\gamma f = 0
$$
\proof Soit $z_0 \in D, z \in D, [z_0, z] \subset D$, alors on definit:
$$
F(z) = \int_{z_0}^z f
$$
Si $h$ est suffisament petit, $[z, z + h] \subset D, [z + h, z_0] \subset D$ alors l'intégrale autour du triangle $z_0, z, z + h$ est nulle. Il s'ensuit que en definissant:
$$
F(z + h) - F(z) = \int_z^{z+h} f = \int_0^1 dt f(z + th) h 
$$
On en deduit si $h \rightarrow 0$ que:
$$
F'(z) = f(z)
$$

\theorem Soit $D$ un ouvert connexe, $f$ analytique, $\gamma_1, \gamma_2$ deux lacets de $D$, homotopes alors:
$$
\int_{\gamma_1} f(z) dz = \int_{\gamma_2} f(z) dz
$$

\corollary Si $D$ est simplement connexe alors l'integrale est nulle pour tout lacet.

\subsection{Indince d'un point par rapport à un lacet.}
\definition Soit $\gamma$ un lacet de $I = [a,b]$ et $w \not\in \gamma(I)$ alors on defini l'indice comme étant:
$$
\text{Ind}_\gamma (w) = \frac{1}{2 \pi i} \int_\gamma \frac{1}{z - w} dz
$$
C'est un entier relatif qui quantifie l'enroulement de $\gamma$ autour de $w$.
\begin{remark}
On a $\text{Ind}_{\gamma_0}(w) = -\text{Ind}_\gamma(w)$ et si deux lacets sont homotopes leur indices coincides.
\prop Si $\gamma$ est un lacet, et $D$ connexe tel que $\gamma(I) \cap D = \emptyset$ alors $w \rightarrow \text{Ind}_\gamma(w)$ est constante.
\end{remark}

\subsection{Formule de Cauchy}
\theorem Soit $D$ simplement connexe, $\gamma : I \to D$ alors pout toute fonction $f$ analytique on a:
$$
\forall w \in D \setminus \gamma(I), 2\pi i f(w) \text{Ind}_\gamma(w) = \int_\gamma dz \frac{f(z)}{z - w}
$$

\subsection{Conditions de Cauchy.}
\theorem Si $f$ est continument derivable dans $D$ ouvert alors $f$ est analytique.

\end{document}