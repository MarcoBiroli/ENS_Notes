\documentclass[10pt,a4paper]{book}
\usepackage[utf8]{inputenc}
\usepackage[english]{babel}
\usepackage{amsmath}
\usepackage{amsfonts}
\usepackage{amssymb}
\usepackage{wrapfig}
\usepackage{mathtools}
\usepackage{graphicx}
\usepackage{cancel}
\usepackage[left=2cm,right=2cm,top=2cm,bottom=2cm]{geometry}
\usepackage{physics}
\usepackage{multicol}
\usepackage{caption}
\usepackage{subcaption}
\author{Marco Biroli}
\title{Mechanics of continuous media.}
\begin{document}
\maketitle
\tableofcontents

\chapter{...}
\section{Beyond linear elasticity.}
At the microscopical level the internal forces are caused by the interactions in between the particles and by an external applied force $f$ where we have: $f \approx \sigma r_0^2$. The interaction potential is usually given by something of the following form:
....

Then for $\frac{r - r_0}{r_0} \ll 1$, $f$ is linear in $r - r_0$ it is the elastic regime, but what happens if we pull harder? Two behaviors are possible:
\subsection{Fragile Materials.}
Fragile materials will stay elastic until a breaking point which happens for deformations that are still quite low $\varepsilon \sim \frac{1}{1000}$. Examples of such materials are glass, ceramic, plexiglas, graphite etc... The atomic model explains quite well this phenomenon from a qualitative point of view. We know that there exists a $\sigma_\text{max}$ such that $\frac{f}{r_0^2} > \sigma_\text{max}$. Knowing $V(r)$ we can therefore deduce $\sigma_\text{max}$. It is determined by the microscopic properties, and a scale reasoning done to estimate the Young modulus $E$ is valid to estimate $\sigma_\text{max}$ as well. For example for glass we have that: $E = 70 \text{ GPa}$ and the theoretical $\sigma_\text{max} = 20 \text{ GPa}$. From which we would deduce a deformation $\varepsilon = \frac{\sigma}{E} \sim 30\%$, which is way too high. In truth the experimental value is $\sigma_\text{max} = 0.1 \text{ GPa}$ which gives a deformation of $\varepsilon = \frac{1}{700} \sim 0.14 \%$. The reason for this considerable difference between theory and reality is that nothing is perfect in nature. Hence the crystal lattice will contain errors which lead to a much lower breaking point. Griffith in 1920 postulated the existence of superficial or volumic fissures where the constraints are concentrated. For example in 2D for an elliptical fissure where we would have $\frac{a}{b} = 10$ then we would get $\sigma_\text{summit} = 20 \sigma_\text{external}$.

\subsubsection{Griffith Criterion.}
When we put a small bar under constraint then the elastic energy will be given by: $\sigma_{ij} \varepsilon_{ij} \cdot \text{volume}$ and in order of magnitude we have $E = \sigma \varepsilon l S$ with $\sigma = E \varepsilon = \frac{\sigma^2}{E} l S$. The existence of a small imperfection of size $a$ will release the constraint on a characteristic length of size $a$. Hence with the defect we have an energy that is freed:
\[
\delta E_\text{elastic} = - \frac{\sigma^2}{E} a^3
\]
However to create the imperfection it costs a certain amount of energy: $\delta E_\text{breaking} = \Gamma a^2$ where $\Gamma$ is the surface tension of the solid and $a^2$ the zone we are breaking. At a microscopical level we have:
\[
\Gamma = \frac{\text{energy}}{\text{surface}} = \frac{\text{rydberg}}{a_0^2} \approx \frac{10^{-19} \text{ J}}{(10^{-10})^2 \text{ m}^2} = 10 \text{ J}.\text{m}^{-2}
\]
So overall when we create an imperfection we have:
\[
\delta E = - \frac{\sigma^2}{E} a^3 + \Gamma a^2
\]
So we get something of the form:
[INCLUDE GRAPHICS]

Where we have $a_c = \frac{\Gamma E}{\sigma^2}$, where defect with size greater than $a_c$ have a tendency to grow. If we consider that rupture happens for $\sigma_\text{break} \sim \frac{E}{100}$ then we get:
\[
a_c = 10^4 \frac{\Gamma}{E} \approx 1 \, \mu\text{m} = \text{ typical defect size}, \Gamma = \frac{\text{micro energy}}{(\text{micro size})^2}, E = \frac{\text{micro energy}}{(\text{micro size})^3}
\]

\subsection{Plastic Materials (ductile).}
[INCLUDE GRAPHIC]
When we augment $\varepsilon$ we pass from an elastic to a plastic regime. When we release the constraint we will get an elastic regime that has the same parameters as the initial one with the exception of the plasticity threshold having become higher. However the material will be permanently deformed. In this regime the deformation can be very significant with respect to the elastic regime (for example of the order of 50\%). The microscopic origin of this is a shifting of the microscopic planes of the solid. So if we pass the plasticity threshold the planes slide, the material gets deformed plasticly, the volume is conserved. It goes back to a shifted equilibrium position when the constraint is released and when it reaches it we find the same elastic properties.


\end{document}