\documentclass[10pt,a4paper]{book}
\usepackage[utf8]{inputenc}
\usepackage[english]{babel}
\usepackage{amsmath}
\usepackage{amsfonts}
\usepackage{amssymb}
\usepackage{wrapfig}
\usepackage{mathtools}
\usepackage{graphicx}
\usepackage{cancel}
\usepackage[left=2cm,right=2cm,top=2cm,bottom=2cm]{geometry}
\usepackage{physics}
\usepackage{multicol}
\usepackage{caption}
\usepackage{subcaption}
\author{Alessandro Pacco}
\title{DM-MMC, 2020}
\begin{document}
\maketitle

\tableofcontents

\section*{De la forme des arbres}


\section*{Méthodes d'analyse complexe pour des problèmes d'élasticité bidimensionelle}

\subsection*{1.a)}
If $f$ is a holomorphic function then it is $C^{\infty}$ on all of its domain. 
Then we have that 
\begin{align*}
f'(z)&=\lim_{|h|\to 0}\frac{f(z+h)-f(z)}{h}=\lim_{h_1\to 0,h_2\to 0}\frac{f_1(z_1+h_1,z_2+h_2)-f_1(z_1,z_2)+i[f_2(z_1+h_1,z_2+h_2)-f_2(z_1,z_2)]}{h_1+ih_2}
\end{align*}
now this limit must be valid both when $h_1=0$ and $h_2\to 0 $ and when $h_1\to 0$ and $h_2=0$. From this it follows that 
\begin{align*}
f'(z)&=\frac{\partial f_1(z_1,z_2)}{\partial z_1}+i\frac{\partial f_2(z_1,z_2}{\partial z_1}\\
f'(z)&=-i\frac{\partial f_1(z_1,z_2)}{\partial z_2}+\frac{\partial f_2(z_1,z_2)}{\partial z_2}
\end{align*} from which the Cauchy conditions follow, i.e.
\begin{align*}
\frac{\partial f_1}{\partial z_1}&=\frac{\partial f_2}{\partial z_2}\\
\frac{\partial f_1}{\partial z_2}&=-\frac{\partial f_2}{\partial z_1}
\end{align*}


\subsection*{1.b)}
Since $f_1$ and $f_2$ are $C^2$, then we can exchange the order of derivation, thus getting
$$\Delta f_1=\partial^2_{z_1}f_1+\partial^2_{z_2} f_1=\partial^2_{z_1z_2}f_2+\partial^2_{z_2z_1}(-f_2)=0$$

similarly for $f_2$ we get that 
$$\Delta f_2=\partial^2_{z_1}f_2+\partial^2{z_2}f_2=\partial^2_{z_1z_2}(-f_1)+\partial^2_{z_2z_1}f_1=0$$

\subsection*{2)}
At equilibrium we have that 
$$\phi+\div(\sigma)=0$$
where $$\div(\sigma)=\begin{pmatrix}
\frac{\partial \sigma_{xx}}{\partial x}+\frac{\partial \sigma_{xy}}{\partial y}+\frac{\partial \sigma_{xz}}{\partial z}\\
\frac{\partial \sigma_{yx}}{\partial x}+\frac{\partial \sigma_{yy}}{\partial y}+\frac{\partial \sigma_{yz}}{\partial z}\\
\frac{\partial \sigma_{zx}}{\partial x}+\frac{\partial \sigma_{zy}}{\partial y}+\frac{\partial \sigma_{zz}}{\partial z}
\end{pmatrix}$$Componentwise and using Einstein's notation we can therefore write: $\phi_i+\partial_j(\sigma_{ij})=0$.
Still with Einstein's notation we have that Hooke's law says that 
$$\sigma_{ij}=2\mu \epsilon_{ij}+\lambda \epsilon_{kk}\delta_{ij}$$
Now it follows that 
\begin{align*}
\partial_j\sigma_{ij}=2\mu\partial_j\epsilon_{ij}+\lambda \partial_i\epsilon_{kk}
\end{align*}
and using the fact that $\epsilon_{kk}=\div(\mathbf{u})$ and $\partial_j\epsilon_{ij}=\frac{1}{2}\partial_j(\partial_j u_i+\partial_i u_j)=\frac{1}{2}\partial_{jj}u_i+\frac{1}{2}\partial_j\partial_i u_j=\frac{1}{2}\laplacian u_i+\frac{1}{2}\partial_j\partial_i u_j$
we get that
$\partial_j\sigma_{ij}=2\mu[ \frac{1}{2}\laplacian u_i+\frac{1}{2}\partial_j\partial_i u_j]+\lambda\partial_i\div(\mathbf{u})$. Finally it follows that
\begin{align*}
\phi_i+\mu\laplacian u_i+\mu\partial_j\partial_i u_j +\lambda\partial_i\div(\mathbf{u})=0\Rightarrow \phi+\mu\laplacian \mathbf{u}+(\mu+\lambda)\grad(\div(\mathbf{u}))=0
\end{align*}



\subsection*{3)}
We have that $\mathbf{u}=(0,0,\omega)$,
from which, using the equation found previously, it follows that 
$$\mu\laplacian(\omega)=0\Rightarrow\partial^2_x\omega+\partial^2_y\omega=0$$
where we used the fact that in this case $\div(\mathbf{u})=0$. So $\omega$ is harmonic

%$\epsilon_{xx}=\epsilon_{xy}=\epsilon_{yx}=\epsilon_{yy}=\epsilon_{zz}=0$ and $\epsilon_{zx}=\epsilon_{xz}=\frac{1}{2}\frac{\partial w}{\partial x}$, $\epsilon_{yz}=\epsilon_{zy}=\frac{1}{2}\frac{\partial w}{\partial y}$. 

\subsection*{4)}
We have that 
$$\sigma=\begin{pmatrix}
0 && 0 && \frac{1}{2}\frac{\partial \omega}{\partial x}\\
0 && 0 && \frac{1}{2}\frac{\partial \omega}{\partial y}\\
\frac{1}{2}\frac{\partial \omega}{\partial x} && \frac{1}{2}\frac{\partial \omega}{\partial y} && 0\\
\end{pmatrix}$$




ì




\end{document}