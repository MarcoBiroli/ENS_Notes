\documentclass[10pt,a4paper]{book}
\usepackage[utf8]{inputenc}
\usepackage[english]{babel}
\usepackage{amsmath}
\usepackage{amsfonts}
\usepackage{amssymb}
\usepackage{wrapfig}
\usepackage{mathtools}
\usepackage{graphicx}
\usepackage{cancel}
\usepackage[left=2cm,right=2cm,top=2cm,bottom=2cm]{geometry}
\usepackage{physics}
\usepackage{multicol}
\usepackage{caption}
\usepackage{subcaption}
\author{Alessandro Pacco}
\title{DM-MMC, 2020}
\begin{document}
\maketitle

\tableofcontents

\section*{De la forme des arbres}
We assume that trees will try to grow as tall as possible whilst staying balanced and not breaking in order to get as much sunlight as possible. Hence the problem is to find what is the maximal height that a cylinder of diameter $2d$ can reach before becoming unstable. We know that the area $A$ of a cross-section of the cylinder is given by $4 \pi d^2$.  We denote by $E$ Young's modulus of the tree and assume it to be constant in all of the tree. Then  denoting by $\rho$ the radius of curvature of the line  that links all the successive centers of the circles building the cylinder. Theory of elasticity given in class tells us that:
\[
M = E I \frac{1}{\rho}
\]
Then using the following coordinates: [INSERT GRAPH]
We can write:
\[
\frac{1}{\rho} = \dv[2]{y}{x}
\]
I.e. our previous equation becomes:
\[
M(P) = EA k^2 \dv[2]{y}{x}
\]
However this is also given by:
\[
M(P) = \lambda A \int_0^x (y' - y) \dd x'
\]
This gives rise to the following differential equation:
\[
EA k^2 \dv[2]{y}{x} = \lambda A \int_0^x (y' - y) \dd x'
\]
Derivating w.r.t. $x$ on both sides and taking $p = \dv{y}{x}$ we get:
\[
E A k^2 \dv[2]{p}{x} = - \lambda A \int_0^x p \dd x' = - \lambda A x p
\]
Which can be re-written as:
\[
x^2 \dv[2]{p}{x} + \frac{\lambda}{E k^2}  x^3 p = 0 
\]
Now doing a change of variable by taking $p = x^{1/2} z$ we get:
\[
x^2 \dv[2]{z}{x} + x \dv{z}{x} + (\frac{\lambda}{E k^2} x^3 - \frac{1}{4}) z = 0
\]
Now taking $x^3 = r^2$ we get:
\[
x^2 \dv[2]{z}{r} + x \dv{z}{r} + \left(\frac{4 \lambda}{9 E k^2} r^2 - \frac{1}{9}\right) z = 0
\]
Which is re-written as:
\[
r^2 \dv[2]{z}{r} + r \dv{z}{r} + (\frac{4 \lambda}{9 E k^2} r^2 - \frac{1}{9}) z = 0
\]
This is a special case of Bessel's differential equation. Hence whe know it's exact solution which is given by:
\[
z = A J_{1/3}(\kappa r) + B J_{-1/3} (\kappa r) \text{ where } \kappa = \sqrt{\frac{4 \lambda}{9 E k^2}}
\]
Which by reversing the variable changes gives as original solution:
\[
p = \sqrt{x} \left(A J_{1/3}(\kappa x^{3/2}) + B J_{-1/3} (\kappa x^{3/2}) \right)
\]
The border condition $\dv{p}{x}\Bigg|_{x = 0} = 0$ gives that $A = 0$. Then at $A$ we must have $p = 0$ and we denote by $h$ the height of the tree then we need to solve:
\[
J_{-1/3}(\kappa h^{3/2}) = 0
\]
We know that the Bessel functions admit roots and call $c$ its first root. Then we get that:
\[
c = \kappa h^{3/2} \Rightarrow h = \left(\frac{9 E k^2 c^2}{4 \lambda}\right)^{1/3}
\]
Numerically we know $c \approx 1.88$ which then gives:
\[
h = 1.52 \left(\frac{9 E}{16 \lambda} \right)^{1/3} d^{2/3} \propto d^{2/3}
\]

\section*{Méthodes d'analyse complexe pour des problèmes d'élasticité bidimensionelle}

\subsection*{1.a)}
If $f$ is a holomorphic function then it is $C^{\infty}$ on all of its domain. 
Then we have that 
\begin{align*}
f'(z)&=\lim_{|h|\to 0}\frac{f(z+h)-f(z)}{h}=\lim_{h_1\to 0,h_2\to 0}\frac{f_1(z_1+h_1,z_2+h_2)-f_1(z_1,z_2)+i[f_2(z_1+h_1,z_2+h_2)-f_2(z_1,z_2)]}{h_1+ih_2}
\end{align*}
now this limit must be valid both when $h_1=0$ and $h_2\to 0 $ and when $h_1\to 0$ and $h_2=0$. From this it follows that 
\begin{align*}
f'(z)&=\frac{\partial f_1(z_1,z_2)}{\partial z_1}+i\frac{\partial f_2(z_1,z_2}{\partial z_1}\\
f'(z)&=-i\frac{\partial f_1(z_1,z_2)}{\partial z_2}+\frac{\partial f_2(z_1,z_2)}{\partial z_2}
\end{align*} from which the Cauchy conditions follow, i.e.
\begin{align*}
\frac{\partial f_1}{\partial z_1}&=\frac{\partial f_2}{\partial z_2}\\
\frac{\partial f_1}{\partial z_2}&=-\frac{\partial f_2}{\partial z_1}
\end{align*}


\subsection*{1.b)}
Since $f_1$ and $f_2$ are $C^2$, then we can exchange the order of derivation, thus getting
$$\Delta f_1=\partial^2_{z_1}f_1+\partial^2_{z_2} f_1=\partial^2_{z_1z_2}f_2+\partial^2_{z_2z_1}(-f_2)=0$$

similarly for $f_2$ we get that 
$$\Delta f_2=\partial^2_{z_1}f_2+\partial^2{z_2}f_2=\partial^2_{z_1z_2}(-f_1)+\partial^2_{z_2z_1}f_1=0$$

\subsection*{2)}
At equilibrium we have that 
$$\phi+\div(\sigma)=0$$
where $$\div(\sigma)=\begin{pmatrix}
\frac{\partial \sigma_{xx}}{\partial x}+\frac{\partial \sigma_{xy}}{\partial y}+\frac{\partial \sigma_{xz}}{\partial z}\\
\frac{\partial \sigma_{yx}}{\partial x}+\frac{\partial \sigma_{yy}}{\partial y}+\frac{\partial \sigma_{yz}}{\partial z}\\
\frac{\partial \sigma_{zx}}{\partial x}+\frac{\partial \sigma_{zy}}{\partial y}+\frac{\partial \sigma_{zz}}{\partial z}
\end{pmatrix}$$Componentwise and using Einstein's notation we can therefore write: $\phi_i+\partial_j(\sigma_{ij})=0$.
Still with Einstein's notation we have that Hooke's law says that 
$$\sigma_{ij}=2\mu \epsilon_{ij}+\lambda \epsilon_{kk}\delta_{ij}$$
Now it follows that 
\begin{align*}
\partial_j\sigma_{ij}=2\mu\partial_j\epsilon_{ij}+\lambda \partial_i\epsilon_{kk}
\end{align*}
and using the fact that $\epsilon_{kk}=\div(\mathbf{u})$ and $\partial_j\epsilon_{ij}=\frac{1}{2}\partial_j(\partial_j u_i+\partial_i u_j)=\frac{1}{2}\partial_{jj}u_i+\frac{1}{2}\partial_j\partial_i u_j=\frac{1}{2}\laplacian u_i+\frac{1}{2}\partial_j\partial_i u_j$
we get that
$\partial_j\sigma_{ij}=2\mu[ \frac{1}{2}\laplacian u_i+\frac{1}{2}\partial_j\partial_i u_j]+\lambda\partial_i\div(\mathbf{u})$. Finally it follows that
\begin{align*}
\phi_i+\mu\laplacian u_i+\mu\partial_j\partial_i u_j +\lambda\partial_i\div(\mathbf{u})=0\Rightarrow \phi+\mu\laplacian \mathbf{u}+(\mu+\lambda)\grad(\div(\mathbf{u}))=0
\end{align*}



\subsection*{3)}
We have that $\mathbf{u}=(0,0,\omega)$,
from which, using the equation found previously, it follows that 
$$\mu\laplacian(\omega)=0\Rightarrow\partial^2_x\omega+\partial^2_y\omega=0$$
where we used the fact that in this case $\div(\mathbf{u})=0$. So $\omega$ is harmonic

%$\epsilon_{xx}=\epsilon_{xy}=\epsilon_{yx}=\epsilon_{yy}=\epsilon_{zz}=0$ and $\epsilon_{zx}=\epsilon_{xz}=\frac{1}{2}\frac{\partial w}{\partial x}$, $\epsilon_{yz}=\epsilon_{zy}=\frac{1}{2}\frac{\partial w}{\partial y}$. 

\subsection*{4)}
We have that 
$$\epsilon=\begin{pmatrix}
0 && 0 && \frac{1}{2}\frac{\partial \omega}{\partial x}\\
0 && 0 && \frac{1}{2}\frac{\partial \omega}{\partial y}\\
\frac{1}{2}\frac{\partial \omega}{\partial x} && \frac{1}{2}\frac{\partial \omega}{\partial y} && 0\\
\end{pmatrix}$$
and from $\sigma_{ij}=2\mu\epsilon_{ij}+\lambda\epsilon_{kk}\delta_{ij}$ we get that 
$$
\sigma=2\mu\begin{pmatrix}
 0 && 0 && \frac{1}{2}\frac{\partial \omega}{\partial x}\\
0 && 0 && \frac{1}{2}\frac{\partial \omega}{\partial y}\\
\frac{1}{2}\frac{\partial \omega}{\partial x} && \frac{1}{2}\frac{\partial \omega}{\partial y} && 0\\
\end{pmatrix}=
\begin{pmatrix}
 0 && 0 && \frac{\partial\text{Im}\Omega}{\partial x}\\
0 && 0 && \frac{\partial\text{Im}\Omega}{\partial y}\\
\frac{\partial\text{Im}\Omega}{\partial x} && \frac{\partial\text{Im}\Omega}{\partial y} && 0\\
\end{pmatrix}$$
Finally we have that $\sigma_{yz}+i\sigma_{xz}=\frac{\partial\text{Im}\Omega}{\partial y}+i\frac{\partial\text{Im}\Omega}{\partial x}=\frac{\partial\text{Re}\Omega}{\partial x}+i\frac{\partial\text{Im}\Omega}{\partial x}=\Omega$.

\subsection*{5)}
From $\log(z)=\ln(|z|)+i\arg(z)$ we get that $\Omega(z)=-iP\ln(|z|)/2\pi+\arg(z)/2\pi$, which implies that the displacement field has its $z$ component $\omega$ given by $\mu\omega(x,y)=-P\ln(\sqrt{x^2+y^2})/2\pi$. Then it follows that $\mu\partial_x\omega=-\frac{P}{2\pi}\frac{x}{x^2+y^2}$ and $\mu\partial_y\omega=-\frac{P}{2\pi}\frac{y}{x^2+y^2}$, and finally
$$\sigma=-\frac{P}{2\pi}\frac{1}{x^2+y^2}\begin{pmatrix}
0 && 0 && x\\
0 && 0 && y\\
x && y && 0
\end{pmatrix}$$
This corresponds to a situation where 

\subsection*{6.a)}
With $$\sigma_l=\begin{pmatrix}
0 && 0 && 0\\
0 && 0 && S\\
0 && S && 0
\end{pmatrix}$$
the behaviour of $\Omega$ should be such that when $|z|$ goes to infinity, then $\partial_y\text{Im}\Omega\to S$ and $\partial_x\text{Im}\Omega\to 0$, which by the Cauchy conditions also implies that 
$\partial_x\text{Re}\Omega\to S$ and $\partial_y\text{Re}\Omega\to 0$.

\subsection*{6.b)}
If we call $z=x+iy$, then we have that $\Omega=S(z-R^2/z)=S\bigg(x-\frac{R^2x}{x^2+y^2}+i\frac{x^2y+R^2y^2+y^3}{x^2+y^2}\bigg)$ from which it follows that $\omega(x,y)=\frac{S}{\mu}\frac{x^2y+R^2y^2+y^3}{x^2+y^2}$. The displacement field is given by $\mathbf{u}=(0,0,\omega)$. Finally we have that 
$$\frac{\partial\text{Im}\Omega}{\partial x}=-S\frac{2R^2xy^2}{(x^2+y^2)^2}$$
$$\frac{\partial\text{Im}\Omega}{\partial y}=S\frac{2x^2y(R^2+y)+x^4+y^4}{(x^2+y^2)^2}$$
so that 
$$\sigma=
\frac{S}{(x^2+y^2)^2}
\begin{pmatrix}
0 && 0 && -2R^2xy^2\\
0 && 0 && 2x^2y(R^2+y)+x^4+y^4\\
-2R^2xy^2 && 2x^2y(R^2+y)+x^4+y^4 && 0
\end{pmatrix}
$$
We have the following behaviors:


\subsection*{6.c)}
At the boundary we need to have that $\sigma\cdot \vec{n}cst$


\subsection*{7.a)}
We have that 
\begin{align*}
\Gamma(z_0)=\frac{a+b}{2}z_0+\frac{a-b}{2z_0}=x_0\frac{a(|z_0|^2+1)+b(|z_0|^2-1)}{2|z_0|^2}+iy_0\frac{a(|z_0|^2-1)+b(|z_0|^2+1)}{2|z_0|^2}
\end{align*}
which implies that 
$$x:=\text{Re}\Gamma=x_0\frac{a(x_0^2+y_0^2+1)+b(x_0^2+y_0^2-1)}{2(x_0^2+y_0^2)}$$
$$y:=\text{Im}\Gamma=y_0\frac{a(x_0^2+y_0^2-1)+b(x_0^2+y_0^2+1)}{2(x_0^2+y_0^2)}$$

Now we want to see where circles of radius $r$ are mapped by $\Gamma$: suppose we consider the circle of radius $r$, i.e. the set of points satisfying $x_0^2+y_0^2=r^2$, then we get that
$$\begin{cases}
x=x_0\big(\frac{a+b}{2}+\frac{a-b}{2r^2}\big)\\
y=y_0\big(\frac{a+b}{2}-\frac{a-b}{2r^2}\big)
\end{cases}\Rightarrow \frac{x^2}{\big(\frac{a+b}{2}+\frac{a-b}{2r^2}\big)^2}+\frac{y^2}{\big(\frac{a+b}{2}-\frac{a-b}{2r^2}\big)^2}=r^2$$
hence the image of a circle of radius $r$ is an ellipse with semi-major and semi-minor axes given respectively by $r\frac{a+b}{2}+\frac{a-b}{2r}$ and $r\frac{a+b}{2}-\frac{a-b}{2r}$. Since the circle of radius $1$ has been excluded from the domain, it follows that the ellipse of semi-major and semi-minor axes given by $a,b$ is not mapped by any circle. In order to conlude that the image of $\Gamma$ is the complex plane without the ellipse of axes $a,b$ we need to prove that this ellipse is not intersected by any other ellipse of the type found above. To prove this we use the fact that if two ellipses with semi-major/semi-minor axes respectively given by $g_1,g_2$ and $f_1,f_2$ are such that $g_1>f_1$ and $g_2>f_2$, then these two ellipses don't intersect. In particular consider a generical ellipse of the type found above, $\frac{x^2}{\big(r\frac{a+b}{2}+\frac{a-b}{2r}\big)^2}+\frac{y^2}{\big(r\frac{a+b}{2}-\frac{a-b}{2r}\big)^2}=1$, mapped by $\Gamma$ from a circle of radius $r>1$; then we have that 
\begin{align*}
&r\frac{a+b}{2}+\frac{a-b}{2r}>a\\
\Leftrightarrow & r^2(a+b)-2ar+(a-b)>0\\
\Leftrightarrow & (r-1)(r-\frac{a-b}{a+b})>0\\
\Leftrightarrow &\text{(True since }r>1)
\end{align*}

and \begin{align*}
&r\frac{a+b}{2}-\frac{a-b}{2r}>b\\
\Leftrightarrow &r^2(a+b)-2rb-(a-b)>0\\
\Leftrightarrow &(r-1)(r-\frac{b-a}{a+b})>0\\
\Leftrightarrow &\text{(True since }r>1)
\end{align*}
instead if $r<1$ we have that 

\subsection*{7.b)}
The question is equivalent to finding $\omega(x,y)$ by knowing $\omega(x_0,y_0)$. In principle we have to find the imaginary part of $\Omega(\Gamma(z_0))$ and then substitute $x_0,y_0$ as a function of $x,y$. 





























\end{document}