\documentclass[10pt,a4paper]{book}
\usepackage[utf8]{inputenc}
\usepackage[english]{babel}
\usepackage{amsmath}
\usepackage{amsfonts}
\usepackage{amssymb}
\usepackage{wrapfig}
\usepackage{mathtools}
\usepackage[makeroom]{cancel}
\usepackage{graphicx}
\usepackage{cancel}
\usepackage[left=2cm,right=2cm,top=2cm,bottom=2cm]{geometry}
\usepackage{physics}
\usepackage{multicol}
\usepackage{caption}
\usepackage{subcaption}
\usepackage{braket} %\braket{a|b|c..}

%main sets:
\newcommand{\Z}{\mathbb{Z}}
\newcommand{\Q}{\mathbb{Q}}
\newcommand{\R}{\mathbb{R}}
\newcommand{\C}{\mathbb{C}}

%math shortcuts
\newcommand{\p}{\partial}
\newcommand{\w}{\omega}
\newcommand{\tf}{\text{TF}}



\author{Alessandro Pacco}
\title{Statistical physics solutions}
\begin{document} 
\maketitle


\chapter*{Chapter 6}
\section*{Exercise 6.3)}




Remark: diffraction and diffusion.

Send a laser on a lens $D1$. We have $A(x)=0$ if the light doens't pass, 1 otherwise. Introduce the angle $\theta$. Then $r(x,z)\approx r_0(z)-x\sin(\theta)$. This approximation is valid only if $x<<d$. The spherical wave coming out of a hole is $E(r)=\frac{E_0}{r}e^{i\frac{2\pi}{\lambda}(r-ct)}$. Then $$E_{tot}(z)=\int dxA(x)\frac{E_0}{r(x,z)}e^{i\frac{2\pi}{\lambda}r(x,z)}\underset{\underset{\text{varie lentement}}{x\to x}}{\approx} \frac{E_0}{r_0(z)}e^{i\frac{2\pi}{\lambda}r_0(z)}\int dxA(x)e^{i2\pi\frac{\sin\theta}{\lambda}x}=\frac{E_0}{r_0(z)}e^{i\frac{2\pi}{\lambda}r_0(z)}TF(A)\frac{\sin\theta}{\lambda}$$
if we mesure the light on the screen, we get that
$$I(z)=|E_{tot}(z)|^2\approx |TF(A)|^2\approx |S|^2$$.


\subsection*{1)}
We have $\rho^{(l)}(\vec{x}_1,\cdots,\vec{x}_l)=\mathbb{P}($particella 1 at $\vec{x}_1\pm d^3\vec{r}\cap\cdots\cap$ particella $l$ at $\vec{x}_l\pm d^3\vec{r}).$
Moreover $E_p=\sum_{i>j} U(\vec{r}_i-\vec{r}_j)$.

\subsection*{2)}
The kinetic energy is given by $E_c=\sum_{i}\frac{p_i^2}{2m}$.

\subsection*{3)}
As always
$$Z=\frac{1}{h^{3N}\cdot N!}\int d\Gamma e^{-\beta (E_c+E_p)(\Gamma)}=\frac{1}{h^{3N}\cdot N!}\bigg(\sqrt{\frac{2\pi m}{\beta}}\bigg)^{3N}\int \prod d^{3}\vec{r}_i e^{-\beta E_p(\Gamma)}=\frac{\lambda^{-3N}}{N!}Q$$
with $\lambda=\sqrt{\frac{h^{2}\beta}{2\pi m}}$

\subsection*{4)}
In order to have a classical description we need to have that (average distance between particles)$>>\lambda$. Hence since the average particle distance is $(V/N)^{1/3}$ we get that we must have$(V/N)^{1/3}=n^{-1/3}>>\lambda$.


\subsection*{5)}
We proved that 
$$\frac{1}{V}g(\vec{r})d^3\vec{r}=\mathbb{P}(\text{ 1 particle in }\vec{r}\pm d^3\vec{r}|\text{1 part in 0})\underset{GP}{=}\mathbb{P}(\text{1 part in }\vec{r}\pm d^3\vec{r})=\frac{d^3\vec{r}}{V}$$
and so $g_{GP}(\vec{r})=1$. 

\subsection*{6)}
Benzene is non-polar, hence it is better than water for the diffraction. Difficulty to obtein some data for small $\vec{r}$? If $\vec{ r}$ is too small with respect to the wavelength we don't get diffraction.\\
Estimation of the diamater:



\subsection*{7)}
For diffraction we need that the $\sigma$ is almost the same as the wavelenght of the incoming light. Hence We have resolution$\sim\sigma\sim 500nm$ since we want $\lambda\sim\sigma$. Hence we need a laser of $\lambda \sim 514  nm$.


\subsection*{8)}

$$n=\frac{\text{number of particles}}{V}=\frac{\text{densite massique}}{\text{masse 1 particule}}=\frac{5\cdot 39\cdot 10^{-2}}{8\cdot 10^{-15}}\approx 10^{19}m^{-3}$$
minimal volume $\sim\sigma^3$.
We have that
$$n\sigma^3=10^{19}(5\cdot 10^{-7})^3=125\cdot 10^{19-21}\approx 1.25\text{ particles/elementar volume}$$
Hence it is not at all dilute, but mainly dense.

\subsection*{9)}

We work on the LHS:

\begin{align*}
\rho^{(2)}=\sum_{i_1\neq i_2}
<\delta(\vec{r}_{i_1}-\vec{x})\delta(\vec{r}_{i_2}-\vec{y})>
&=\frac{1}{Z}\int\prod_{i=1}^n\frac{d\vec{r}_i d\vec{p}_i}{h^3}e^{-\beta H(..)}
\cdot \sum_{i_1\neq i_2} 
\underset{Q3}{=}\frac{1}{Q_N}\sum_{i_1\neq i_2}\int\prod_{i=1}^N d^3\vec{r}_i e^{-\beta E_p\{r_i\}}
\delta(\vec{r}_{i_1}-\vec{x})\delta(\vec{r}_{i_2}-\vec{y})\\
&=
\frac{1}{Q_N}N(N-1)\int\prod_{i=1}^N d^3\vec{r}_i
e^{-\beta\sum_{i<j}U(r_i-r_j)}\delta(\vec{r}_1-\vec{x})\delta(\vec{r}_2-\vec{y})\\
&=\frac{N(N-1)}{Q_N}
\int\prod_{i=3}^Nd^3\vec{r}_i
e^{-\beta\big[U(x-y)+\sum_{i=3}^N(U(x-r_i)+U(y-r_i))+\sum_{i>j=3}^N U(r_i-r_j)\big]}
\end{align*}

Now we attack the gradient 

\begin{align*}
\vec{\nabla}_{\vec{x}}\rho^{(2)}(x,y)=
\frac{N(N-1)}{Q_N}\int\prod_{i=3}^Nd^3\vec{r}_i(-\beta\nabla U(x-y)-\beta\sum_{i=3}^N\nabla U(x-r_i))\cdot e^{-\beta(..)}
=\frac{N(N-1)}{Q_N}(-\beta\nabla U(x-y))\int\prod_{i=3}^Nd^3r_ie^{-\beta(..)}-\frac{N(N-1)}{Q_N}\beta\int\prod_{i=3}^Nd^3r_i\big[\sum_{i=3}^N\nabla U(x-r_i)\big]e^{-\beta E_p}
\end{align*}

Then we introduce $\sum_{i=3}^Nf(r_i)=\int dzf(z)\sum_{i=3}^N\delta(z-r_i)$. With that decomposition we get

$$\nabla_{vec{x}}\rho^{(2)}(x,y)=-\beta(\nabla U(x-y))\rho^{(2)}(x,y)-\beta\int dz\nabla U(x-z)\frac{N(N-1)}{Q_N\int \prod_{i=3}^Ndr_ie^{-\beta E_p(x,y,r_{i>3})}}\sum_{i=3}^N\delta(r_i-z)
$$

where $<>$ is the canonical average.

\subsection{10)}


Born-Green is general.

$\nabla\rho^2\approx f(\rho^2,\rho^3)$ and $\nabla \rho^3=f(\rho^3,\rho^4)$, it doens't stop. 

$$\rho^{(l)}(\vec{x}_1,\vec{x}_2,..,\vec{x}_n)=\sum_{i_1\neq..\neq i_l}<\delta(\vec{x}_{i_1}-vec{x}_{1})..\delta(\vec{x}_{i_l}-\vec{x}_l)>\underset{dilue}{\approx}\sum<\delta(\vec{x}_{i_1}-\vec{x}_{i_1})..\delta(..)>\approx n\sigma^3)^l$$
IN practice, if we suppose dilute then $n\sigma^3<<1$ implies that $\rho^{(1)}>>\rho^{(2)}>>...$.

In our case we have 

$$merda da fare$$

By symmetry $\rho^{(2)}(x,y)=\rho^{(2)}(x-y)$. If we rewrite Born-Green at order 2 we get

$$\frac{d}{dr}\rho^{(2)}=-\beta(\frac{d}{dr}U(r))\rho^{(2)}(r)\Rightarrow \rho^{(2)}(r)=Ae^{-\beta U(r)}$$ We determine $A$ with $\rho^{(2)}\underset{r\to\infty}{=}n^{-2}$ and $U(r)\underset{r\to\infty}{\to 0}$, so that $A=n^{-2}$.


\subsection*{11)}


on paper

\subsection*{12)}


Computing the TF of $g(\vec{r})-1$. Hors $g(\vec{r})=g(r,\xcancel{\theta},\xcancel{\phi})$.

$$TF(g-1)(\vec{q})=\int d^3\vec{r} e^{i\vec{q}\cdot\vec{r}}(g(\vec{r})-1)=(coord spher z aligne \vec{q})=\int 4\pi r^2dr\int_0^{\pi}d\theta \sin\theta e^{iqr\cos\theta}(g(r)-1)
=4\pi\int r^2dr(g(r)-1)\int_{-1}^1 du\cdot e^{iqru} u=\cos(\theta)$$
with $\int(merda -1 1)=2\frac{\sin(qr)}{qr}$.

Hence $$TF(g-1)(q)=\frac{8\pi}{q}\int dr\cdot r(g(r)-1)\sin(qr)=\frac{4\pi}{q^3}\bigg((e^{\beta\epsilon}-1)(\sin\alpha\sigma q-\alpha q\cos \alpha\sigma q)+e^{\beta\epsilon}(\sigma q \cos(\sigma q)-\sin(\sigma q))$$
pseudoperiod $\sim\frac{2\pi}{\sigma},\frac{2\pi}{\alpha\sigma}$. $\rho^2\sim$ period $\sigma,\alpha\sigma$. Then we get 
$$S(q)=1+nTF(g-1)(q)$$ and so $S(0)=1+n+4\pi(-(e^{\beta\epsilon})(\alpha\sigma)^3+e^{\beta\epsilon}\sigma^3)$ since $\sin(x)-x\cos(x)=x-x^3/6-x(1-x^2/2)=x^3/3+o(x^3)$ (S behaves well in 0).

\subsection*{14)}
Fig. 6.3, we can read $S(0)$ and know $n,\beta$: for $\alpha$, we take $o(1)$, $\alpha=2$ for example. THen $\beta\epsilon_{exp}\underset{\alpha^3>>1}{\approx}\frac{1-S(0)}{4\pi n\sigma^3/3}\alpha^{-3}\approx 10^{-2}$, $\frac{\epsilon}{k_B}\sim 3K$ (on the paper 70K). $\epsilon_{exp}<<k_BT$ implies that the thermic agitation dominates.

\subsection*{15)}

Experimentally, we have that $S(q)\to_{\tf^{-1}exact }g(r)\sim e^{-\beta U(r)}(approximation diluee)$. We can determine experimentally the approxiative interactions $U(r)$
$$U_{exp}(r)=-k_BT\ln(g_{exp}(r))$$

FIg 6.3 $\Phi(r)=-\ln g_{exp}(r)$. 





\chapter*{Chapter 11}
\section*{11.4}
\subsection*{1)}
Call $q$ the number of defects in the chain. Then at each time we encounter a defect, we have to add a term $-1$, whereas inside the defect we only add $+1$ terms, because the particles have same spin. In simple words any meeting of two defect steals a spin link. Hence the hamiltonian will be given by 
$$H=-J((N-q)-(q-1))=-J(N-2q+1)$$

\subsection*{2)}
The degeneracy of an energetic level is given by the number of possible configurations that lead to that energy. In particular since an energy is given by $q$, the number of possiblities that we can get that energy will be given by the number of possibilities to choose $q$ defects. This is given by the number of possibilities to put $q-1$ sticks in $N-1$ holes, i.e. 
${N-1}\choose{q-1}$, multiplied by $2$, because for any defect configuration we can either start with up or with down.
The canonical partition function will be given by 
\begin{align*}
Z&=
\sum_{q=1}^N 2\binom{N-1}{q-1} e^{\beta J(N-2q+1)}=\sum_{q=0}^{N-1}2\binom{N-1}{q}e^{\beta J(N-1-2q)}=2e^{\beta J(N-1)}\sum_{q=0}^{N-1}\binom{N-1}{q}(e^{-2\beta J})^2\\
&=2e^{\beta J(N-1)}(1+e^{-2\beta J})^{N-1}=2^N\cosh^{N-1}(\beta J)
\end{align*}


\subsection*{3)}
Let's fix the number of defects $q$. 
Then the energy is fixed at $H=-J(N-2q+1)$ and the associated partition function is given by 
$$Z_q=2\binom{N-1}{q-1}e^{\beta J(N-2q+1)}$$
The associated free energy is then given by
\begin{align*}
F_q&=-k_BT\log(Z_q)=-k_BT\log(2\binom{N-1}{q-1}e^{\beta J(N-2q+1)})
=-k_BT\bigg(\log(2)+\log(\binom{N-1}{q-1})+\beta J(N-2q+1)\bigg)\\
&\approx -k_BT\bigg(\log(2)+\beta J(N-2q+1)+(N-1)\log(N-1)-(q-1)\log(q-1)-(N-q)\log(N-q)\bigg)
\end{align*}


\subsection*{4)}
What the fucking hell do you want me to do?



\subsection*{5)}

We have that
\begin{align*}
\langle q\rangle&=\frac{\sum_{\{\sigma_i\}}q_{\{\sigma_i\}}e^{\beta J(N-2q+1)}}{Z}=\frac{e^{\beta J(N+1)}}{-2J}\frac{\p_{\beta} \sum e^{-2\beta J q}}{Z}=-\frac{1}{2J}\frac{\p}{\p\beta}\log(\frac{Z}{e^{\beta J(N+1)}})\\
&=-\frac{1}{2J}\frac{\p}{\p\beta}\bigg(\log(Z)-\beta J(N+1)\bigg)=\frac{N+1}{2}-\frac{1}{2J}\frac{\p}{\p\beta}\bigg(N\log(2)+(N-1)\log(\cosh\beta J)\bigg)\\
&=\frac{N+1}{2}-\frac{1}{2J}(N-1)\frac{\sinh(\beta J)J}{\cosh(\beta J)}=\frac{N+1}{2}-(N-1)\frac{\tanh(\beta J)}{2}
\end{align*}

\subsection*{6)}
We can define $d$ as $d=\frac{N}{\langle q\rangle} $
so that 
$$d=\frac{N}{\frac{N+1}{2}-(N-1)\frac{\tanh(\beta J)}{2}}$$

\subsection*{7)}
The hamiltonian now is 
$$H=-\sum_{i=1}^{N-1}J_i\sigma_i\sigma_{i+1}$$
We will suppose that any configuration of the type $\pm J_1\pm J_2\pm\cdots\pm J_{N-1}$ is unique and can be obtained with only one configuration of the $\sigma_i$. This way there is a bijection between the possible sums of the form above and the choice of the $\sigma_i$ (up to changing the initial value of $\sigma_i$ which automatically determines all the other values of the $\sigma_i$), therefore we will take the $J_i$s with sign. Hence we get that
\begin{align*}
Z&=\sum_{J_i=\pm |J_i|} 2e^{-\beta H}=\sum_{J_i=\pm|J_i|}e^{\beta\sum_{i=1}^{N-1}J_i}=\sum_{J_1=\pm|J_1|}e^{\beta J_1}\ldots \sum_{J_{N-1}=\pm|J_{N-1}|}e^{\beta J_{N-1}}\\
&=2(2\cosh(\beta |J_1|))\cdots (2\cosh(\beta |J_{N-1}|))=2^{N}\cosh(\beta |J_1|)\cdots\cosh(\beta |J_{N-1}|)
\end{align*}

\subsection*{8)}
We have that
\begin{align*}
\langle \sigma_i\sigma_{i+1}\rangle &=\frac{\sum_{\{J_k\}}\sigma_i\sigma_{i+1}2e^{\beta\sum_{k=1}^{N-1}J_k}}{Z}=2\frac{\sum_{J_1=\pm|J_1|}e^{\beta J_1}\cdots \sum_{J_i=\pm|J_i|}\text{sgn}(J_i)e^{\beta J_i}\cdots \sum_{J_{N-1}=\pm|J_{N-1}|}e^{\beta J_{N-1}}}{Z}\\
&=\frac{2(2\cosh(\beta |J_1|)\cdots (2\sinh(\beta |J_i|))\cdots (2\cosh(\beta |J_{N-1}|)}{Z}=\tanh(\beta |J_i|)
\end{align*}
More easily we could have seen that $\langle \sigma_i\sigma_{i+1}\rangle =\frac{1}{Z\beta}\frac{\p Z}{\p J_i}$ (with the old notation in which we keep $J_i$ with its original sign and we multiply it by $\sigma_i\sigma_{i+1}$).

\subsection*{9)}
Let's get back to the original notation since here we saw it becomes easier. We have:
\begin{align*}
\langle \sigma_i\sigma_{i+2}\rangle&=\langle \sigma_i\sigma_{i+1}\sigma_{i+1}\sigma_{i+2}\rangle =\frac{\sum_{\{\sigma_i\}}(\sigma_i\sigma_{i+1})(\sigma_{i+1}\sigma_{i+2})2e^{\beta\sum_{k=1}^{N-1}\sigma_k\sigma_{k+1}J_k}}{Z}=\frac{1}{Z\beta^2}\frac{\p^2 Z}{\p J_i\p J_{i+1}}\\
&=\tanh(\beta J_i)\tanh(\beta J_{i+1})
\end{align*}

Then for $J_i=J$ we get 
$$\langle \sigma_i\sigma_{i+1}\rangle =\tanh^2(\beta J)$$


\subsection*{10)}

We have that
\begin{align*}
\langle\sigma_i\sigma_{i+r}\rangle=\langle\sigma_i\sigma_{i+1}\ldots\sigma_{i+r-1}\sigma_{i+r}\rangle=\frac{1}{Z\beta^r}\frac{\p ^r Z}{\p J_i\ldots \p J_{i+r}}=\tanh(\beta J_i)\ldots\tanh(\beta J_{i+r})
\end{align*}
and by using $J_i=J$ we finally get
$\langle \sigma_i\sigma_{i+r}\rangle =(\tanh(\beta J))^r$. 

Per trovare $g(r)$ manca $\langle \sigma\rangle^2$ (WTF??)

\subsection*{11)}
By using $g(r)=(\tanh\beta J)^r$ (which I am not sure being true wtf lepre mela puttana) then we get that 
$$g(r)=e^{-r/\xi}\Leftrightarrow \xi=-\frac{1}{\ln(\tanh\beta J)}$$
For $N$ big we have that 
$$d\approx \frac{2}{1-\tanh(\beta J)}$$
and for $\beta J>>1$ we have
$$\xi\approx -\frac{1}{\tanh\beta J-1}=\frac{1}{1-\tanh\beta J}$$
hence we can directly see the direct proportionality between $d$ and $\xi$. 

\subsection*{12)}

The proportionality factor is $2$. 

\subsection*{13)}
Suppose that we have a finite temperature $T$ with interactions at short distance. Then if there was a ferromagnetic phase, then there wouldn't be lost of information through the chain, i.e. the first spin would be influencing all the others ( which is what would happen at zero temperature), however we see that the cprrelation function goes to zero as $r$ goes to infinity, which means that the at high distance two spins are completely unrelated.









\end{document}