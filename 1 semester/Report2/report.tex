\documentclass[10pt,a4paper]{article}
\usepackage[utf8]{inputenc}
\usepackage[english]{babel}
\usepackage{amsmath}
\usepackage{amsfonts}
\usepackage{amssymb}
\usepackage{graphicx}
\usepackage[left=2cm,right=2cm,top=2cm,bottom=2cm]{geometry}
\author{Marco Biroli \& Alessandro Pacco}
\title{Report}
\begin{document}
\maketitle

Van der Waals solids were already very widely know. However the revolution in graphene was the ability to extract a single layer of graphene. The 

- Bloch theorem.

- Rich for electronics due to flatland first order dependence on cristal lattice.

- Need to describe wavefunction with a spinnor.

- Interlace graphene with hexagonal boron nitride -> good 2d van der waals hetero strucutre with confinement of the order of 0.3nm

- magic of graphene is that the electrons are strongly decoupled form the lattice.

\end{document}