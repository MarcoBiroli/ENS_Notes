\documentclass[10pt,a4paper]{article}
\usepackage[utf8]{inputenc}
\usepackage[english]{babel}
\usepackage{amsmath}
\usepackage{amsfonts}
\usepackage{amssymb}
\usepackage{physics}
\usepackage[left=2cm,right=2cm,top=2cm,bottom=2cm]{geometry}
\author{Marco Biroli}
\title{Mécanique Analytique}
\begin{document}
\maketitle

\section{Principe de Fermat et applications.}
\subsection{}
\subsubsection{}
On a:
\[
\mathcal{L} = (AMA') = n \cdot d(A, M) + n' \cdot d(M, A') = n \sqrt{x^2 + y_0^2} + n'\sqrt{(x_0 - x)^2 + y_0^2}
\]
\subsubsection{}
Par consequent:
\[
\pdv{\mathcal{L}}{x} = x\frac{n}{\sqrt{x^2 + y_0^2}} + (x - x_0)\frac{n'}{\sqrt{(x_0 - x)^2 + y_0^2}}
\]
Résolvant pour $\pdv{\mathcal{L}}{x} = 0$ on obtient:
\[
n \frac{x}{\sqrt{x^2 + y_0^2}} = n' \frac{(x_0 - x)}{\sqrt{(x_0 - x)^2 + y_0^2}} \Leftrightarrow n \sin i = n' \sin i'
\]
\subsubsection{}
Ceci correspond au principe de moindre action.

\subsubsection{}
Si on autorize $z \neq 0$ on aurait immediatement:
\[
\pdv{\mathcal{L}}{z} = 2 n z  + 2 n' z = z(2n + 2n') = 0 \Rightarrow z = 0 
\]

\subsection{}
L'équation du dipotre est donnée par:
\[
(x-R)^2 + y^2 + z^2 = R^2
\]
Pour obtenir l'approximation parabolloide on doit mettre les termes d'un coté et effectuer une expansion de Taylor:
\[
x - R = \pm \sqrt{R^2 - y^2 - z^2} \Rightarrow x = R - R\sqrt{1 - \frac{y^2 + z^2}{R^2}} = R - R\left(1 + \frac{y^2 + z^2}{2R^2}+ \mathcal{O}(\epsilon^3) \right) \approx \frac{y^2 + z^2}{2R}
\]

\subsection{}
On travaille d'abord sur le premier segment, le deuxième suit immediatement:
\[
(AM) = n \sqrt{(|x_A| - \frac{y^2}{2R})^2 + (y_A - y)^2} = n \sqrt{(|x_A|^2 - \frac{R}{2} \rho^2)^2 + (y_A - R \rho)^2}
\]
Où on prend $\rho = \frac{y}{R} \ll 1$ maintenant en faisant une expansion de Taylor en $\rho$ on obtient:
\[
(AM) = n \sqrt{x_A^2+y_A^2}-\frac{n  R
   y_A}{\sqrt{x_A^2+y_A^2}}\rho-\frac{\left(n \left(-R^2 x_A^2+R
   |x_A|^3+R |x_A| y_A^2\right)\right)}{2
   \left(x_A^2+y_A^2\right)^{3/2}} \rho^2 +O\left(\rho ^3\right)
\]
Meme si l'énoncé ne l'explicite pas nous allons assumer que $|y_A| \ll |x_A|$ car sinon les calculs ne donnent pas la bonne reponse. Par consequent on obtient (on se rappelle du fait que $x_A < 0$):
\[
\kappa_A = n \sqrt{|x_A|^2 + y_A^2} = n\left(-x_A - \frac{y_A^2}{2 x_A}\right) + \mathcal{O}\left(\left|\frac{y_A}{x_A}\right|^2\right) \approx -n x_A - \frac{1}{2} \nu_A y_A^2 
\]
De même on a (on obtient un terme $R$ au denominateur en remplacant $\rho$ par $y/R$):
\[
\beta_A = \frac{n y_A}{\sqrt{x_A^2 + y_A^2}} = \frac{n \frac{y_A}{|x_A|}}{\sqrt{1 + \frac{y_A^2}{x_A^2}}} = n\frac{y_A}{|x_A|} + \mathcal{O}\left(\left(\frac{y_A}{|x_A|}\right)^3\right) \approx -  \nu_A y_A
\]
Et finalement:
\[
-\frac{1}{2}\alpha_A = \frac{\left(n \left(-R^2 x_A^2+R
   |x_A|^3+R |x_A| y_A^2\right)\right)}{2 R^2
   \left(x_A^2+y_A^2\right)^{3/2}} = \frac{nR - \frac{n R^2}{y_A} \frac{y_A}{|x_A|} + nR \frac{y_A^2}{x_A^2}}{2 R^2(1 + \frac{y_A^2}{x_A^2})^{3/2}} = \frac{n}{2R} - \frac{n}{2 y_A} \frac{y_A}{|x_A|} - \frac{n}{4 R}  \frac{y_A^2}{x_A^2} + \mathcal{O}\left(\left(\frac{y_A}{|x_A|}\right)^3\right)
\]
Mais le terme:
\[
\left|\frac{n}{4 R}\frac{y_A^2}{x_A^2}\right| < \left|\frac{n}{4} \frac{y_A}{R} \frac{y_A^2}{x_A^2}\right| = \mathcal{O}\left(\left| \frac{y_A}{x_A}\right|^3\right)
\]
On a donc:
\[
\alpha_A = \frac{n}{R} +\frac{n}{|x_A|} = \frac{n}{R} - \nu_A
\]
De manière identique on obtient:
\[
\kappa_{A'} = n' x_{A'} + \frac{1}{2}\nu_{A'} y_{A'}^2, \quad \beta_{A'} = \nu_{A'} y_{A'}, \quad \alpha_{A'} = -\frac{n'}{R} + \nu_{A'}
\]
Et puisque $\mathcal{L} = (AMA') = (AM) + (MA')$ on a $\alpha = \alpha_{A} + \alpha_{A'}, \beta = \beta_A + \beta_{A'}, \kappa = \kappa_A + \kappa_{A'}$, cqfd.

\subsection{}
Le principe de Fermat nous enonce que $\mathcal{L}$ doit être stationnaire. Il nous faut donc que:
\[
\dv{\mathcal{L}}{y}\Bigg|_{y_0} = 0 \Rightarrow  y_0 \alpha - \beta = 0 \Rightarrow y_0 = \frac{\beta}{\alpha} 
\]

\subsection{}
Dans le cadre de nos approximations $\mathcal{L}$ est une parabole et par consequent l'extremum est un minimum si et seulement si $\alpha > 0$ soit:
\begin{align*}
\frac{n'}{x_{A'}} - \frac{n}{x_A} - \frac{n' - n}{R} > 0 &\Leftrightarrow \frac{n' x_A R - n x_{A'} R - n' x_A x_{A'} + n x_{A} x_{A'}}{x_{A'}x_A R} > 0\\
&\Leftrightarrow -n'|x_A|R - n x_{A'} R + n' |x_A| x_{A'} - n |x_A| x_{A'} < 0
\end{align*}

\subsection{}
La condition de stigmatisme imposerait que $\mathcal{L}(y)$ soit constante afin que quel que soit $y$ le chemin est minimal. Par consequent le stigmatisme impose $\alpha = \beta = 0$. On a donc (on note par simplicite $y_A = y, y_{A'} = y'$, de même pour les autres variables):
\[
\begin{cases}
\nu y - \nu' y' = 0\\
\nu' - \nu - \frac{n' - n}{R} = 0
\end{cases}
\Leftrightarrow
\begin{cases}
y' = \frac{\nu R y}{R\nu + n' - n}\\
\nu' = \nu + \frac{n' - n}{R}
\end{cases}
\]
\subsection{}
\subsection{}
On a donc:
\[
\delta\mathcal{L} = \int_{s_1}^{s_2} \left( \grad n \cdot u - \dv{}{s}\left(n(\vec{r}) \vec{u}\right) \vec{u} \right) \cdot \delta \vec{r} \cdot \dd \vec{r}
\]
Par consequent on obtient:
\[
\grad n = \dv{}{s} \left(n(\vec{r})\vec{u}\right) = \dv{}{s}\left(n(\vec{r})\dv{\vec{r}}{s}\right)
\]
\subsection{}
En introdusiant $\dd s = n(\vec{r})\dd a$ on obtient:
\[
\dv[2]{}{a} \vec{r} = n(\vec{r})\grad n(\vec{r}) \Leftrightarrow \dv{}{a} \left( n(r) \vec{u}\right) = \grad(\frac{1}{2}n(\vec{r})^2)
\]
On remarque que cette equation est similaire a la deuxième loi de Newton pour la mécanique du point:
\[
m \dv{}{t} \vec{p} = \vec{F} = \grad V
\]
La quantité de mouvement à pour equivalent optique $n(\vec{r}) \vec{u}$, par consequent le moment cinétique est donnée par $n(\vec{r}) \vec{r} \times \vec{u}$. Et finalement l'énergie potentielle est donnée par $\frac{1}{2}n(\vec{r})^2$. 

\subsection{}
Si l'indice optique ne depend que de $r$ on a:
\[
\grad \frac{1}{2}n(r)^2 = \frac{1}{2}\begin{pmatrix}
2 n(r) \dd_r n(r)\\
0\\
0
\end{pmatrix}
\]
Pour utiliser l'analogie mecanique on pose:
\[
\mathcal{L} = \frac{1}{2}(\dot{r}^2 + (r\dot{\varphi})^2 + \dot{z}^2) - n(r) \dv{n(r)}{r}
\]
On note immediatement que $\pdv{\mathcal{L}}{\varphi} = \pdv{\mathcal{L}}{z} = 0$ et avec les equations d'Euler Lagrange on a:
\[
\pdv{\mathcal{L}}{\dot{z}} = cst = m \dot{z} = p_z \quad \text{ et } \quad \pdv{\mathcal{L}}{\dot{\varphi}} = cst = m r^2 \dot{\varphi} = r^2 p_\varphi
\]
En remplacant cela par notre probleme on obtient immediatement:
\[
n(r) \dv{z}{s} = cst = \lambda \quad \text{ et } \quad r^2 n(r) \dv{\varphi}{s} = cst = \mu
\]
La loi d'optique qui se cache derriere la premiere equation est une loi de Snell-Descarte infinitésimale.

\subsection{}
\subsubsection{}
On a:
\[
\left(\dv{r}{z}\right)^2 = \left(\dv{s}{z}\right)^2 - 1 = \frac{n(r)^2}{\lambda^2} - 1 = \frac{n(r)^2 - \lambda^2}{\lambda^2}
\]
\subsubsection{}
Par la question précédente on deduit qu'il nous faut que $\frac{n^2 - \lambda^2}{\lambda^2} \geq 0$. Pour $r$ contenu dans une région de $[-a, a]$. Par le theorème de la valeur moyenne il nous faut donc:
\[
\frac{n_1^2 - \lambda^2}{\lambda^2} \geq 0 \quad \text{ et } \quad \frac{n_2^2 - \lambda^2}{\lambda^2} \leq 0 
\] 
On obtient donc immediatement que:
\[
n_2 \leq \lambda \leq n_1
\]
\subsection{}
En injectant la solution dans l'équation on obtient:
\[
r_0^2 \Omega^2 \sin^2(\Omega z) = \frac{n^2 - \lambda^2}{\lambda^2} = \frac{n_1^2 (1 - 8 \Delta \frac{r_0^2 \cos^2 (\Omega z)}{a^2}) - \lambda^2}{\lambda^2}
\]
Que l'on peut ré-écrire:
\[
r_0^2 \Omega^2(\sin^2 (\Omega z) + \cos^2 (\Omega z)) = \frac{n_1^2 - \lambda^2}{\lambda^2} \Rightarrow r_0^2 = \frac{n_1^2 - \lambda^2}{\lambda^2 \Omega^2}
\]

\section{}
\subsection{}
Les équations vérifiées par $x$ et $y$ sont:
\[
\begin{cases}
\dv{}{t}\pdv{L}{\dot{x}} - \pdv{L}{x} = 0\\
\dv{}{t}\pdv{L}{\dot{y}} - \pdv{L}{y} = 0
\end{cases}
\Leftrightarrow
\begin{cases}
m\ddot{y} - 2\lambda \dot{y} + ky = 0\\
m\ddot{x} + 2\lambda \dot{x} + kx = 0
\end{cases}
\]
On a donc:
\[
x(t) = c \exp(-t\left(\frac{\lambda}{m} + \frac{\sqrt{\lambda^2 - km}}{m}\right)) + c' \exp(t\left( -\frac{\lambda}{m} + \frac{\sqrt{\lambda^2 - km}}{m} \right))
\]
Et:
\[
y(t) = c \exp(t\left(\frac{\lambda}{m} + \frac{\sqrt{\lambda^2 - km}}{m}\right)) + c' \exp(t\left(\frac{\lambda}{m} - \frac{\sqrt{\lambda^2 - km}}{m} \right))
\]

\subsection{}
Le moments conjugués sont donc:
\[
p_x = \pdv{\mathcal{L}}{\dot{x}} = m\dot{y} - \lambda y \quad \text{ et } \quad p_y = \pdv{\mathcal{L}}{\dot{y}} = m\dot{x} + \lambda x
\]

\subsection{}
On a donc:
\begin{align*}
H &= p_x \dot{x} + p_y \dot{y} - \mathcal{L} = (m\dot{y} - \lambda y) \dot{x} + (m\dot{x} + \lambda x) \dot{y} - m \dot{x} \dot{y} + \lambda(y\dot{x} - x\dot{y}) + kxy\\
&= m\dot{x}\dot{y} + kxy = (p_x + \lambda y)(p_y - \lambda x) + k x y
\end{align*}
Puisque les moments conjugués sont définis par des équations scléronomes on a que $H = E$.


\section{}
\subsection{}
On a l'équation suivante:
\[
a \pdv{\rho(\vec{x})}{t} + \frac{3}{\nu^2} \pdv[2]{\rho(\vec{x})}{t} - \laplacian \rho (\vec{x}) = 0
\]
En multipliant a gauche et à droite par $e^{i\vec{q}\cdot \vec{r}}$ et en integrant sur $\mathbb{R}^3$ on peut passer dans l'espace de Fourier et on obtient:
\[
a \pdv{\hat{\rho}(\vec{q}, t)}{t} + \frac{3}{\nu^2} \pdv[2]{\hat{\rho}(\vec{q}, t)}{t} + \vec{q}^2 \hat{\rho}(\vec{q}, t) = 0
\]
Maintenant on peut fixer un quelquonque $\vec{q}$ et on écrit $\hat{\rho}(\vec{q}, t) = u(t)$ et on à donc:
\[
a \dot{u} + \frac{3}{\nu^2} \ddot{u} + \vec{q}^2 u = 0
\]
Si on écrit maintenant cette equation pour le terme réel et imaginaire on obtient immediatement un system similaire a la question précedente. On en deduit qu'on peut modeliser ce problème par la Lagrangienne:
\[
\mathcal{L} = \frac{3}{\nu^2} \pdv{\Re u}{t} \pdv{\Im u}{t} - \frac{a}{2}\left(\Re u \pdv{\Im u}{t} + \Im u \pdv{\Re u}{t}\right) + \vec{q}^2 \Re u \Im u
\]
\subsection{}
En injectant la solution on obtient:
\[
-a (i\omega)\rho + \frac{3}{\nu^2} \omega^2 \rho - k^2 \rho = 0 \Leftrightarrow k = \pm\sqrt{3\frac{\omega^2}{\nu^2} - i a w}
\]
Ceci est similaire a la propagation d'une onde dans un milieu dissipatif.

\section{}
\subsection{}
Si $\psi$ est solution soit $\varphi(\vec{r}, t) = \psi(\vec{r}, -t)$ alors:
\[
i\hbar \pdv{\varphi}{t} = - \frac{h^2}{2m}\pdv[2]{\varphi}{x} + V \varphi \Leftrightarrow - i\hbar \pdv{\varphi}{t} = -\frac{h^2}{2m} \pdv[2]{\varphi}{x} + V \varphi
\]
On remarque que cela est exactement:
\[
\left(i\hbar \pdv{\psi}{t} \right)^* = \left(-\frac{\hbar^2}{2m}\pdv[2]{\psi}{x} + V \psi\right)^* \Leftrightarrow - i \hbar \pdv{\psi^*}{t} = - \frac{\hbar^2}{2m}\pdv[2]{\psi^*}{x} + V\psi^*
\]
On en déduit donc que $\psi^* = \varphi$. On remarque donc que le temps est assimilé à l'ordre absolu sur le droite complexe en mecanique quantique.

\subsection{}
De manière similaire a la question précedente on se propose de passer en espace de Fourier et on obtient:
\[
i \hbar \pdv{\psi}{t} = -\frac{\hbar^2}{2m}\laplacian \psi + V \psi \stackrel{\text{TF}}{\Longrightarrow} i \hbar \pdv{\hat{\psi}}{t} = - \frac{\hbar^2}{2m} \vec{q}^2 \hat{\psi} + \hat{V} \circledast \hat{\psi} 
\]
Par extension soit $M$ la matrice de dimension infinie contenant $\frac{1}{i\hbar}\hat{V}$ sur chaque ligne et soit $W$ la matrice de dimension infinie et diagonale contenant $\frac{i\hbar}{2m} \vec{q}^2$ sur la diagonale. On re-ecrit alors le systeme precedent comme:
\[
\pdv{\hat{\psi}}{t} = (W + M) \hat{\psi}
\]

\end{document}