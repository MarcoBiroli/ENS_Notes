\documentclass[10pt,a4paper]{book}
\usepackage[utf8]{inputenc}
\usepackage[english]{babel}
\usepackage{amsmath}
\usepackage{amsfonts}
\usepackage{amssymb}
\usepackage{wrapfig}
\usepackage{mathtools}
\usepackage[makeroom]{cancel}
\usepackage{physics}
\usepackage{graphicx}
\usepackage{cancel}
\usepackage[left=2cm,right=2cm,top=2cm,bottom=2cm]{geometry}
\usepackage{physics}
\usepackage{multicol}
\usepackage{caption}
\usepackage{stmaryrd}
\usepackage{subcaption}
\usepackage{braket} %\braket{a|b|c..}

%main sets:
\newcommand{\Z}{\mathbb{Z}}
\newcommand{\Q}{\mathbb{Q}}
\newcommand{\R}{\mathbb{R}}
\newcommand{\C}{\mathbb{C}}

%math shortcuts
\newcommand{\p}{\partial}
\newcommand{\w}{\omega}
\newcommand{\tf}{\text{TF}}

\DeclarePairedDelimiter{\ceil}{\lceil}{\rceil}



\author{Alessandro Pacco \& Marco Biroli}
\title{DM Quantum Mechanics 2019}
\begin{document} 
\maketitle

\section*{Problem 1}


\subsection*{1)}

The evolution operator is 
$$\hat{U}(t,0)=e^{-\frac{i}{\hbar}\hat{H}t}=\sum_{k=0}^{\infty} \frac{(-\frac{i}{\hbar}t)^k}{k!}\hat{H}^k$$


\subsection*{2)}

We have that since the Hamiltonian $\hat{H}=\frac{\hat{P}^2}{2m}+U(\hat{X})+V(\hat{X})=\hat{H_0}+V(\hat{X})$ is time independent then:
$$\ket{\psi(\tau)}=\hat{U}(\tau,0)\ket{\psi_0}=\sum_{k=0}^{\infty}\frac{(\frac{-i}{\hbar}\tau)^k}{k!}(\hat{H}_0+\hat{V})^k\ket{\psi_0}\underset{\text{ordre }1}{\approx} \ket{\psi_0}-\frac{i}{\hbar}\tau(\hat{H_0}+V_0\cos(q\hat{X}))\ket{\psi_0}=(1-\frac{i}{\hbar}\tau( E_0+V_0\cos(q\hat{X}))\ket{\psi_0}$$

\subsection*{3)}
We know that:
\[
\bra{x} \ket{\vec{p}} = \frac{e^{-i \vec{p} \cdot \vec{x}}}{\sqrt{2 \pi}^3} 
\]
Then we get that:
\[
\bra{x} \cos q \hat{X} \ket{\vec{p}} = \cos(qx) \bra{x}\ket{\vec{p}} = \frac{e^{i \vec{q} \vec{x}} + e^{-i \vec{q} \vec{x}}}{2} \frac{e^{-i \vec{p} \vec{x}}}{\sqrt{2 \pi}^3} = \frac{1}{2} \left[  \frac{e^{-i (\vec{p} - \vec{q}) \vec{x}}}{\sqrt{2 \pi}^3} + \frac{e^{-i(\vec{p} + \vec{q}) \vec{x}}}{\sqrt{2 \pi}^3} \right] = \frac{1}{2} (\bra{x}\ket{\vec{p} - \vec{q}} + \bra{x} \ket{\vec{p} + \vec{q}})
\]
Since this holds for all $\bra{x}$  we get that:
\[
\cos (q \hat{X}) \ket{p} = \frac{1}{2} \ket{\vec{p} - \vec{q}} + \frac{1}{2} \ket{\vec{p} + \vec{q}}
\]

\subsection*{4)}
Then we have that:
\begin{align*}
\tilde{\psi}(\vec{p}, \tau) &= \bra{\vec{p}}\ket{\psi(\tau)} = \bra{\vec{p}}(1 - \frac{i}{\hbar} \tau (E_0 + V_0 \cos(q \hat{X})) \ket{\psi_0} \\
&= \tilde{\psi}(\vec{p}, 0) (1 - \frac{i}{\hbar}\tau E_0) + \frac{i}{\hbar} \tau V_0 \left( \frac{1}{2}(\ket{\vec{p} - \vec{q}} + \ket{\vec{p} + \vec{q}}) \right)^\dagger \ket{\psi_0} \\
&= \tilde{\psi}(\vec{p}, 0) \left(1 - \frac{i}{\hbar}\left( \tau E_0 + \frac{\tau V_0}{2} (\tilde{\psi}(\vec{p} - \vec{q}, 0) + \tilde{\psi}(\vec{p} + \vec{q}, 0)) \right) \right) 
\end{align*}

\subsection*{5)}
\subsubsection*{a)}
From the uncertainty principle we have that $\Delta x \Delta p \sim \frac{\hbar}{2}$ hence we obtain that:
\[
\Delta \tilde{\psi_0} \sim \frac{\hbar}{2 \Delta x}
\]

\subsubsection*{b)}
From question (4) we see that at time $\tau$ the wavepacket is at a superposition of the wavepacket in $\vec{p}$, $\vec{p} - \vec{q}$ and $\vec{p} + \vec{q}$. Hence it means that wave packet is expanding and splitting in 3 separate peaks.

\subsection*{6)}
If we know let the wavepacket free the Hamiltonian becomes:
\[
\mathcal{H} = \frac{\hat{p}^2}{2m}
\]
Then we have that:
\[
i\hbar \partial_t \tilde{\psi}(\vec{p}, t) = \mathcal{H} \tilde{\psi}(\vec{p}, t) = \frac{p^2}{2m} \tilde{\psi}(\vec{p}, t) \Rightarrow \tilde{\psi}(\vec{p}, t) = A \exp(i\hbar\frac{2m}{p^2}t)
\]
Now using question (4) we have that:
\[
A = \tilde{\psi}(\vec{p}, \tau)\exp(-i\hbar \frac{2m}{p^2}\tau)
\]
Now we can find the position wavefunction by taking the Fourier transform:
\[
\psi(x, t) =  \int_{\mathbb{R}^3} \dd^3 \vec{p}\, \frac{e^{-i \vec{p} \vec{x}}}{\sqrt{2 \pi}^3} \tilde{\psi}(\vec{p}, t)
\]
We obtain that the space wavefunction will be composed of three peaks that will separate over time.

\section*{Problem 2}

\subsection*{1)}
\subsubsection*{a)}
We have that 
\begin{align*}
E(\lambda+\delta\lambda)&=\braket{\psi(\lambda+\delta\lambda)|\hat{H}(\lambda+\delta\lambda)|\psi(\lambda+\delta\lambda)}=\braket{\psi(\lambda)+\frac{\p\psi}{\p\lambda}(\lambda)\delta\lambda|\hat{H}(\lambda)+\frac{\p\hat{H}}{\p\lambda}(\lambda)\delta\lambda|\psi(\lambda)+\frac{\p\psi}{\p\lambda}(\lambda)\delta\lambda}\\
&=E(\lambda)+\delta\lambda\braket{\psi(\lambda)|\frac{\p\hat{H}}{\p\lambda}(\lambda)|\psi(\lambda)}+(\text{ terms in second order of }\delta\lambda)\Rightarrow \frac{dE}{d\lambda}=\bigg\langle\frac{\p\hat{H}}{\p\lambda}\bigg\rangle
\end{align*}

\subsubsection*{b)}
We have that
$$\frac{\p E_{n',l}}{\p e^2}=-2m_ee^{^2}/2\hbar^2(l+n'+1)^2$$
and 
$$\frac{\p E_{n',l}}{\p l}=2m_ee^4/2\hbar^2(l+n'+1)^3$$
Hence from
$$\frac{\p \hat{h}_l}{\p e^2}=-\frac{1}{r}
$$
it follows thanks to the Hellmann-Feynmann theorem that 
$$\bigg\langle \frac{1}{r}\bigg\rangle =-\bigg\langle \frac{\p \hat{h}_l}{\p e^2}\bigg\rangle =\frac{2m_ee^2}{2\hbar^2(l+n'+1)^2}=\frac{m_ee^2}{\hbar^2(l+n'+1)^2}$$ Similarly we find that 
$$\frac{\p\hat{h}_l}{\p l}=\frac{\hbar^2(2l+1)}{2m_e}\frac{1}{r^2}$$
so that by the Hellmann-Feynmann theorem it follows that 
$$\bigg\langle \frac{1}{r^2}\bigg\rangle = \frac{2m_e}{\hbar^2(2l+1)}\frac{2m_ee^4}{2\hbar^2(l+n'+1)^3}=\frac{2m_e^2e^4}{\hbar^4(2l+1)(l+n'+1)^3}$$
Then we recall that $a_0=\frac{\hbar^2}{m_ee^2}$ we find that
\begin{align*}
&\bigg\langle\frac{1}{r}\bigg\rangle
=\frac{1}{a_0(l+n'+1)^2}\\
&\bigg\langle \frac{1}{r^2}\bigg\rangle =\frac{2}{a_0^2(2l+1)(l+n'+1)^3}
\end{align*}

\subsubsection*{c)}
From the way that $u$ has been defined when solving for the eigenstates of a particle in a central potential, we found that $\hat{h}_l	\ket{u_{n',l}}=E_{n',l}\ket{u_{n',l}}$, and in particular $\bra{u_{n',l}}\hat{h}_l=E_{n',l}\bra{u_{n',l}}$. Finally we obtain that 
$$\braket{u_{n',l}|\hat{A}\hat{h}_l-\hat{h}_l\hat{A}|u_{n',l}}=E_{n',l}\braket{u_{n',l}|\hat{A}-\hat{A}|u_{n',l}}=0$$

\subsubsection*{d)}
We have
$$[\hat{p}_r,\hat{p}_r^2]=0
$$

$$[\hat{p}_r,1/r]\psi=-i\hbar \frac{\p}{\p r}\bigg(\frac{\psi}{r}\bigg)+\frac{1}{r}i\hbar\frac{\p \psi}{\p r}=i\hbar\bigg(\frac{1}{r}\frac{\p\psi}{\p r}+\frac{\psi}{r^2}-\frac{1}{r}\frac{\p\psi}{\p r}\bigg)=i\hbar\frac{\psi}{r^2}$$

$$[\hat{p}_r,1/r^2]\psi=-i\hbar \frac{\p}{\p r}\bigg(\frac{\psi}{r^2}\bigg)+\frac{1}{r^2}i\hbar\frac{\p \psi}{\p r}=i\hbar\bigg(\frac{1}{r^2}\frac{\p\psi}{\p r}-\frac{\p}{\p r}\bigg(\frac{\psi}{r^2}\bigg)\bigg)=2i\hbar\frac{\psi}{r^3}$$
from which it follows that 
$$[\hat{p}_r,\hat{h}_l]=2i\hbar\frac{\hbar^2l(l+1)}{2m_er^3}-\frac{i\hbar e^2}{r^2}=i\frac{\hbar^3l(l+1)}{m_er^3}-i\frac{\hbar e^2}{r^2}.$$
Then from the previous question we have that 
$$\langle [\hat{p}_r,\hat{h}_l\rangle=0\Rightarrow\frac{\hbar^2l(l+1)}{m_e}\bigg\langle\frac{1}{r^3}\bigg\rangle-e^2\bigg\langle\frac{1}{r^2}\bigg\rangle=0\Rightarrow l(l+1)\bigg\langle \frac{\hbar^2}{m_er^3}\bigg\rangle=e^2\bigg\langle\frac{1}{r}\bigg\rangle.$$
Finally form this result we get that 
$$\bigg\langle\frac{1}{r^3}\bigg\rangle =\frac{e^2m_e}{\hbar^2l(l+1)}\bigg\langle\frac{1}{r^2}\bigg\rangle=\frac{2}{a_0^3l(l+1)(2l+1)n^3}.$$

\subsubsection*{e)}
We have the condition that $u_{n',l}(0)=0$ (i.e. we have that $u_l\underset{r\to0}{\sim}Cr^{l+1}$) so that 


\subsection*{2)}

\subsubsection{a)}

From Gauss's differential law we have:
\[
\laplacian V = -\frac{\rho}{\varepsilon_0} = 0 \quad (\text{ out of the nucleus } )
\]

\subsubsection*{b)}
Far away from the nucleus we make the approximation that we can use spherical symmetry. Hence the Legendre expansion of the potential gives the following:
\[
V(r, \theta) = \sum_{l = 0}^{+\infty} (A_l r^l + B_l r^{-(l+1)}) P_l(\cos\theta)
\]
Now setting the boundary condition: $V(r = +\infty) = 0$ we get that $A_l = 0, \quad \forall l \in \mathbb{N}$. Since we assumed $r \gg 1$ we also keep only the terms up to order $2$ in $\frac{1}{r}$. Then plugging in the expression of the Legendre polynomials we get:
\[
V(r, \theta) = B_0 r^{-1} + B_1 r^{-2} \cos \theta + B_2 r^{-3} \underbrace{\frac{1}{2} (3 \cos^2\theta - 1)}_{Y_{2,0} \sqrt{\frac{4 \pi}{5}}}
\]
Now using the orthonormality of the Legendre polynomials we get that:
\[
B_1 r^{-2} \propto \int_{0}^{\pi} V(r, \theta) \cos \theta \dd \theta
\]
Now noticing that we have given an ellipsoid oriented along $z$ we have plane symmetry on $(Oxy)$ hence we get:
\[
V(r, \theta) = V(r, -\theta)
\]
This then means that the integral above cancels which forces $B_1 = 0$. Now the first order term is well known from electrostatics and the second order terms can be simplified to a spherical harmonic as shown previously. Hence we get:
\[
V(r, \theta) = \underbrace{-\frac{e^2}{r}}_{V_0(r)} + \underbrace{B_2 \sqrt{\frac{4\pi}{5}} r^{-3} Y_{2,0}(\theta)}_{V_1(r, \theta)} 
\]

\subsubsection*{c)}
We have that the energy is given by:
\[
E_n = \frac{-m_e e^4}{2\hbar (l + n' + 1)^2} = \frac{-m_e e^4}{2\hbar n^2}
\]
Then we have the following constraints: $ l \in \llbracket 0, n-1 \rrbracket$ and $ m \in \llbracket -l , l \rrbracket$. Hence the degeneracy is given by:
\[
\sum_{l = 0}^{n - 1} \sum_{m = - l}^l 1 = \sum_{l = 0}^{n - 1} (2l + 1) = 2 n\frac{n-1}{2} + n = n^2 - n + n = n^2 
\]


\subsubsection*{d)}
We have that at the first order for a degenerate state of energy, perturbation theory gives the following law:
\[
\mathbf{M}^{(n)} \cdot \vec{c} = \delta E^{(n)} \vec{c}
\]
Where the coefficients of the matrix $\mathbf{M}$ are given by the following expression:
\begin{align*}
M^{(n)}_{\ell, m_\ell, \ell', m_{\ell'}'} &= \bra{n, \ell, m_\ell} \hat{V} \ket{n, \ell', m_{\ell'}'} = \int_0^{+\infty} \dd r \int_0^{2\pi} \int_0^\pi r^2 \sin(\theta) \dd \theta \dd \varphi \frac{u_{n, \ell}(r)}{r} Y_{\ell}^{m_\ell} (\Omega)^* g \frac{Y_{2}^0(\Omega)}{r^3} \frac{u_{n, \ell'}(r)}{r} Y_{\ell'}^{m_{\ell'}'} (\Omega) \\
&= \int_0^{+\infty} \dd r \frac{u_{n, \ell}(r) u_{n, \ell'}(r)}{r^3} \int \dd^2 \Omega \, Y_\ell^{m_\ell} (\Omega)^* Y_2^0 (\Omega) Y_{\ell'}^{m_{\ell'}'} (\Omega)
\end{align*}



\subsubsection*{e)}
We have the following expression for the spherical harmonics:
\[
Y_\ell^{m_\ell} (\theta, \varphi) = N e^{i m_\ell \varphi } P_\ell^{m_\ell} (\cos \theta) 
\]
Now we want to show that the spherical harmonics are orthogonal for the $L_2$ inner product. By simple computation we have:
\begin{align*}
\int_0^\pi \int_0^{2\pi} Y_{\ell'}^{m_{\ell'}'}(\theta,\varphi)^* Y_{\ell}^{m_\ell} (\theta, \varphi) \dd \varphi \dd \theta &= \int_0^{\pi} N^2 P_{\ell'}(\cos \theta) P_\ell (\cos \theta) \int_0^{2\pi} \exp(i \varphi(m_\ell - m_{\ell'}')) \dd \varphi \\
&\propto \int_0^{pi} P_{\ell'}(\cos \theta) P_\ell (\cos\theta) \delta_{\ell, \ell'} \dd \theta 
\end{align*}
Now applying the orthogonality of the Legendre polynomials we get that:
\[
\int_0^\pi \int_0^{2\pi} Y_{\ell'}^{m_{\ell'}'}(\theta,\varphi)^* Y_{\ell}^{m_\ell} (\theta, \varphi) \dd \varphi \dd \theta \propto \delta_{\ell, \ell'} \delta_{m_\ell, m_{\ell'}'}
\]
In our case since $Y_{2}^0(\theta, \varphi) = Y_2^0(\theta)$ we get a special case of the above result:
\[
\int_{0}^\pi \int_0^{2\pi} Y_{\ell}^{m_\ell}(\theta, \varphi)^* Y_2^0 (\theta) Y_{\ell'}^{m_{\ell'}'}(\theta, \varphi) \dd \varphi \dd \theta = \int_0^{\pi} Y_2^0(\theta) A \delta_{m_\ell, m_{\ell'}'}
\]

\subsubsection*{f)}
We know:
\[
Y_0^0(\theta, \varphi) = \frac{1}{\sqrt{4 \pi}}
\]
Then we can take $Y_{\ell'}^{m_{\ell'}'}$ out of the integral and result follows immediately from the orthogonality proven in question (e).


\subsubsection*{g)}
From question (f) we get immediately that $M^{(1)}_{\ell, m_\ell, \ell', m_{\ell'}'}$ will be 0. Hence the fundamental state is unperturbed. The matrix $\mathbf{M}$ is given by:
\begin{align*}
\mathbf{M^{(2)}} = 
\begin{pmatrix}
M^{(2)}_{0, 0, 0, 0} & M^{(2)}_{0, 0, 1, 0} & M^{(2)}_{0, 0, 1, 1} & M^{(2)}_{0, 0, 1, -1}\\
M^{(2)}_{1, 0, 0, 0} & M^{(2)}_{1, 0, 1, 0} & M^{(2)}_{1, 0, 1, 1} & M^{(2)}_{1, 0, 1, -1} \\
M^{(2)}_{1, 1, 0, 0} & M^{(2)}_{1, 1, 1, 0} & M^{(2)}_{1, 1, 1, 1} & M^{(2)}_{1, 1, 1, -1} \\
M^{(2)}_{1, -1, 0, 0} & M^{(2)}_{1, -1, 1, 0} & M^{(2)}_{1, -1, 1, 1} & M^{(2)}_{1, -1, 1, -1}
\end{pmatrix}
\end{align*}
Now using questions (e) and (f) we can simplify as follows:
\begin{align*}
\mathbf{M^{(2)}} = 
\begin{pmatrix}
0 & M^{(2)}_{0, 0, 1, 0} & 0 & 0\\
0 & M^{(2)}_{1, 0, 1, 0} & 0 & 0 \\
0 & 0 & M^{(2)}_{1, 1, 1, 1} & 0 \\
0 & 0 & 0 & M^{(2)}_{1, -1, 1, -1}
\end{pmatrix}
\end{align*}
Now since we have that $Y_{1}^{0}(\Omega) = Y_{1}^0 (\Omega)^*$ we can use the same reasoning used in question (e) to justify that $M^{(2)}_{0,0,1,0} = 0$. Then we have that:
\begin{align*}
M_{1,0,1,0}^{(2)} = g\int_0^{+\infty} \dd r \frac{u_{2, 1}(r) u_{2, 1}(r)}{r^3} \int \dd^2 \Omega \, Y_1^{0} (\Omega)^* Y_2^0 (\Omega) Y_{1}^{0} (\Omega)
\end{align*}
Putting the above in a computation tool and taking:
\[
\ket{n, \ell, m_\ell} = 2^{l+1} e^{-\frac{r}{a n}} \sqrt{\frac{a^3 (-l+n-1)!}{n^4 (l+n)!}} \left(\frac{r}{a
   n}\right)^l L_{-l+n-1}^{2 l+1}\left(\frac{2 r}{a n}\right) Y_l^m(\theta ,\phi )
\]
Yields the correct result.

\section*{Problem 3}


\subsection*{1)}

Since all the operators involving the $z$-component are not composed with operators including the $x,y$-components, then we can write $\hat{H}=\hat{H}_{x,y}\otimes I+I\otimes\hat{H}_z$. Therefore if we call $\ket{\psi_{\perp}}$( reps. $\ket{\psi_z}$) the eigenstates of $\hat{H}_z$ (resp. $\hat{H}_{x,y}$ ) with energy $E_z$ (resp. $E_{\perp}$), then the eigenstate $\ket{\psi}=\ket{\psi_{\perp}}\otimes\ket{\psi_z}$ satisfies
$$\hat{H}\ket{\psi}=(\hat{H}_{x,y}\otimes I)\ket{\psi}+(I\otimes\hat{H}_z)\ket{\psi}=E_{\perp}\ket{\psi}+E_z\ket{\psi}=(E_{\perp}+E_{z})\ket{\psi}$$
Hence $\ket{\psi}$ is an eigenstate of $\hat{H}$ with eigenvalues $E_z+E_{\perp}$. 
Finally we get 
$$\psi(x,y,z)=\braket{x,y,z|\psi}=\bra{x}\otimes\bra{y}\bra{z}\ket{\psi_{\perp}}\otimes\ket{\psi_z}=\psi_{\perp}(x,y)\psi_z(z)$$

\subsection*{2)}
Since the operators including the component $x$ are not composed with those involving the component $y$, we can write that $\hat{H}_{x,y}=\hat{H}_x\otimes I+I\otimes\hat{H}_y$, with $\hat{H}_x$ and $\hat{H}_y$ the Hamiltonians for two harmonic oscillators of pulsations $\omega_{\perp}$. Hence the spectrum of $\hat{H}_x$ (resp. $\hat{H}_y$) is given by $E_{n_x}=\hbar \omega_{\perp}(n_x+1/2)$ (reps. $E_{n_y}=\hbar\omega_{\perp}(n_y+1/2)$). If we denote by $\ket{\psi_{n_x}}$ and $\ket{\psi_{n_y}}$ the two respective eigenstates, then we get that
$$\hat{H}_{x,y}\ket{\psi_{n_x}}\otimes\ket{\psi_{n_y}}=(\hat{H}_x\otimes I)\ket{\psi_{n_x}}\otimes\ket{\psi_{n_y}}+(I\otimes\hat{H}_y)\ket{\psi_{n_x}}\otimes\ket{\psi_{n_y}}=(E_{n_x}+E_{n_y})\ket{\psi_{n_x}}\otimes\ket{\psi_{n_y}}$$
i.e. the eigenstates of $\hat{H}_{x,y}$ are the $\ket{\psi_{n_x}}\otimes\ket{\psi_{n_y}}$, with eigenvalues $E_n=(E_{n_x}+E_{n_y})=\hbar\omega_{\perp}(n_x+n_y+1)=\hbar\omega_{\perp}(n+1)$. Then for a given $n$ (which we recall being a non-negative integer) the degeneracy of the eigenspace associated to $E_n$ is $n+1$ (since we can take $n_x\in [0,n]$ with $n_x\in\mathbb{N}$ and $n_y$ is automatically chosen).

\subsection*{3)}

We have that in any isolated system the total angular momentum and the Hamiltonian commute (which is equivalent to the conservation of the total angular momentum). Hence since the rotation is happening only on the $z$ axis, we have that $$[\hat{H},\hat{L}_z]=0$$
where we have that both $\hat{L}_z$ and $\hat{H}_{x,y}$ act on the same Hilbert space.


\subsection*{4)}

\subsubsection*{a)}

We have that 
$$\hat{a}_x=\frac{\hat{x}\sqrt{\frac{m\omega_{\perp}}{\hbar}}+i\hat{p_x}\frac{1}{\sqrt{m\hbar\omega_{\perp}}}}{\sqrt{2}}=\frac{\frac{\hat{x}}{a_0}+i\frac{\hat{p}_x}{p_0}}{\sqrt{2}}$$

$$\hat{a}_y=\frac{\hat{y}\sqrt{\frac{m\omega_{\perp}}{\hbar}}+i\hat{p_y}\frac{1}{\sqrt{m\hbar\omega_{\perp}}}}{\sqrt{2}}=\frac{\frac{\hat{y}}{a_0}+i\frac{\hat{p}_y}{p_0}}{\sqrt{2}}$$

$$\hat{a}_{x,y}=\hat{a}_x\otimes\hat{a}_y$$

\subsubsection*{b)}
From $[\hat{a}_x,\hat{a}^{\dagger}_x]=1$ (and same for $y$), we get that 
$$[\hat{a}_{x,y},\hat{a}^{\dagger}_{x,y}]=[\hat{a}_x\otimes\hat{a}_y,\hat{a}^{\dagger}_x\otimes \hat{a}^{\dagger}_y]=1$$

\subsubsection*{c)}

From $\hat{x}=\frac{a_0}{\sqrt{2}}(\hat{a}_x+\hat{a}^{\dagger}_x)$ and $\hat{p}_x=\frac{p_0}{i\sqrt{2}}(\hat{a}_x-\hat{a}^{\dagger}_x)$ (and similarly for $y$) we get that
\begin{align*}
\hat{L}_z&=\hat{x}\otimes \hat{p}_y-\hat{y}\otimes\hat{p}_x\\
&=\frac{a_0}{\sqrt{2}}(\hat{a}_x+\hat{a}^{\dagger}_x)\otimes \frac{p_0}{i\sqrt{2}}(\hat{a}_y-\hat{a}^{\dagger}_y)-\frac{a_0}{\sqrt{2}}(\hat{a}_y+\hat{a}^{\dagger}_y)\otimes \frac{p_0}{i\sqrt{2}}(\hat{a}_x-\hat{a}^{\dagger}_x)\\
&=\frac{a_0p_0}{2i}(\hat{a}_x\otimes \hat{a}_y-\hat{a}_x\otimes \hat{a}_y^{\dagger}+\hat{a}_x^{\dagger}\otimes \hat{a}_y-\hat{a}_x^{\dagger}\otimes \hat{a}_y^{\dagger}-\hat{a}_y\otimes \hat{a}_x+\hat{a}_y\otimes \hat{a}_x^{\dagger}-\hat{a}_y^{\dagger}\otimes \hat{a}_x+\hat{a}_y^{\dagger}\otimes \hat{a}_x^{\dagger})\\
&=i\hbar(\hat{a}_x\otimes \hat{a}_y^{\dagger}-\hat{a}^{\dagger}_x\otimes \hat{a}_y)
\end{align*}
where we simplified the terms $\hat{a}_x\otimes \hat{a}_y$ with $\hat{a}_y\otimes \hat{a}_x$ since both $\hat{x},\hat{p}_x$ commute with $\hat{y},\hat{p}_y$ (same for $\hat{a}_x^{\dagger}\otimes \hat{a}_y^{\dagger}$).
Similarly using the fact that we already know the Hamiltonian for a one dimensional harmonic oscillator, we get
\begin{align*}
\hat{H}_{x,y}=\hat{H}_x\otimes I+I\otimes\hat{H}_y=\hbar\omega_{\perp}(\hat{N}_x\otimes I+\frac{1}{2}I\otimes I)+\hbar\omega_{\perp}(I\otimes \hat{N}_y+\frac{1}{2} I\otimes I)=\hbar\omega_{\perp}(\hat{a}_x^{\dagger}\hat{a}_x\otimes I+I\otimes \hat{a}_y^{\dagger}\hat{a}_y +I\otimes I)
\end{align*}

\subsubsection{d)}
By using again the fact that $\hat{x},\hat{p}_x$ commute with $\hat{y},\hat{p}_y$ we get 
$$[\hat{a}_+,\hat{a}_+^{\dagger}]=\frac{1}{2}[\hat{a}_x+i\hat{a}_y,\hat{a}_x^{\dagger}-i\hat{a}^{\dagger}_y]=\frac{1}{2}([\hat{a}_x,\hat{a}_x^{\dagger}]+[\hat{a}_y,\hat{a}_y^{\dagger}])=1$$

$$[\hat{a}_-,\hat{a}_-^{\dagger}]=\frac{1}{2}[\hat{a}_x-i\hat{a}_y,\hat{a}_x^{\dagger}+i\hat{a}^{\dagger}_y]=\frac{1}{2}([\hat{a}_x,\hat{a}_x^{\dagger}]+[\hat{a}_y,\hat{a}_y^{\dagger}])=1$$


$$[\hat{a}_+,\hat{a}_-^{\dagger}]=\frac{1}{2}[\hat{a}_x+i\hat{a}_y,\hat{a}_x^{\dagger}+i\hat{a}^{\dagger}_y]=\frac{1}{2}([\hat{a}_x,\hat{a}_x^{\dagger}]-[\hat{a}_y,\hat{a}_y^{\dagger}]=0$$

$$[\hat{a}_-,\hat{a}_+^{\dagger}]=\frac{1}{2}[\hat{a}_x-i\hat{a}_y,\hat{a}_x^{\dagger}-i\hat{a}^{\dagger}_y]=\frac{1}{2}([\hat{a}_x,\hat{a}_x^{\dagger}]-[\hat{a}_y,\hat{a}_y^{\dagger}])=0$$

$$[\hat{a}_+,\hat{a}_-]=\frac{1}{2}[\hat{a}_x+i\hat{a}_y,\hat{a}_x-i\hat{a}_y]=0\Rightarrow [\hat{a}_+^{\dagger},\hat{a}_-^{\dagger}]=0$$
From now on I will omit the tensor products not to make the text too lengthy.

\subsubsection*{e)}
If two operators commute then they are diagonalizable in the same basis. Here we have, by decomposing the commutator with the well known relations to decompose a commutator which contains a  composition of operators

$$[\hat{N}_{+},\hat{N}_-]=[\hat{a}_+^{\dagger}\hat{a}_+,\hat{a}_-^{\dagger}\hat{a}_-]=\hat{a}_+^{\dagger}[\hat{a}_+,\hat{a}_-^{\dagger}]\hat{a}_-+
\hat{a}_+^{\dagger}\hat{a}_-^{\dagger}[\hat{a}_+,\hat{a}_-]+[\hat{a}_+^{\dagger},\hat{a}_-^{\dagger}]\hat{a}_-\hat{a}_++\hat{a}_-^{\dagger}[\hat{a}_+^{\dagger},\hat{a}_-]\hat{a}_+=0
$$
Hence they are diagonalizable in the same basis.
Now notice that from $[\hat{a}_x,\hat{a}_x^{\dagger}]=1$ (and same for $y$) and from the fact that $\hat{a}_x$ commutes with $\hat{a}_y$, we have $[\hat{a}_{\pm},\hat{a}_{\pm}^{\dagger}]=\frac{1}{2}[\hat{a}_x\pm i\hat{a}_y,\hat{a}_x^{\dagger}\mp i\hat{a}_y^{\dagger}]=1$.
Therefore if we denote by $\ket{n_{\sigma}}$ ($\sigma\in\{+,-\}$) an eigenstate of $\hat{N}_{\sigma}$ of eigenvalue $n_{\sigma}$, we have that:
$$\hat{N}_{\sigma}\hat{a}_{\sigma}\ket{n_{\sigma}}=\hat{a}_{\sigma}^{\dagger}\hat{a}_{\sigma}\hat{a}_{\sigma}\ket{n_{\sigma}}=(\hat{a}_{\sigma}\hat{a}_{\sigma}^{\dagger}\hat{a}_{\sigma}-1)\ket{n_{\sigma}}=(n_{\sigma}-1)\hat{a}_{\sigma}\ket{n_{\sigma}}$$

$$\hat{N}_{\sigma}\hat{a}_{\sigma}^{\dagger}\ket{n_{\sigma}}=\hat{a}_{\sigma}^{\dagger}\hat{a}_{\sigma}\hat{a}_{\sigma}^{\dagger}\ket{n_{\sigma}}=\hat{a}_{\sigma}^{\dagger}(\hat{a}_{\sigma}^{\dagger}\hat{a}_{\sigma}+1)\ket{n_{\sigma}}=(n_{\sigma}+1)\hat{a}_{\sigma}^{\dagger}\ket{n_{\sigma}}$$

which means that if $\hat{a}_{\sigma}\ket{n_{\sigma}}\neq 0$ then it is an eigenstate of $\hat{N}_{\sigma}$ of eigenvalue $(n_{\sigma}-1)$ and that $\hat{a}_{\sigma}^{\dagger}\ket{n_{\sigma}}$ is an eigenstate of $\hat{N}_{\sigma}$ of eigenvalue $(n_{\sigma}+1)$. Moreover we have that the eigenvalues of $\hat{N}_{\sigma}$ are non-negative, indeed we have that $0\leq ||\hat{a}_{\sigma}\ket{n_{\sigma}}||^2=\braket{n_{\sigma}|\hat{a}_{\sigma}^{\dagger}\hat{a}_{\sigma}|n_{\sigma}}=n_{\sigma}$. Therefore if any of the eigenvalues of $\hat{N}_{\sigma}$, say $s_{\sigma}>0$, were not a natural integer, then we could apply $\hat{a}_{\sigma}$ on $\ket{s_{\sigma}}$ $\ceil*{s_{\sigma}}$ times, getting then a negative eigenvalue, which we know being impossible. Hence the spectrum of $\hat{N}_{\sigma}$ is included in $\mathbb{N}$.


\subsubsection{f)}

We have that $\hat{N}_-=\hat{a}_-^{\dagger}\otimes \hat{a}_-=\frac{1}{2}(\hat{a}_x^{\dagger}+i\hat{a}_y^{\dagger})\otimes(\hat{a}_x-i\hat{a}_y)=\frac{1}{2}(\hat{a}_x^{\dagger}\hat{a}_x-i\hat{a}_x^{\dagger}\hat{a}_y+i\hat{a}_y^{\dagger}\hat{a}_x+\hat{a}_y^{\dagger}\hat{a}_y)$ and
$\hat{N}_+=\hat{a}_+^{\dagger}\otimes\hat{a}_+=\frac{1}{2}(\hat{a}_x^{\dagger}-i\hat{a}_y^{\dagger})\otimes (\hat{a}_x+i\hat{a}_y)=\frac{1}{2}(\hat{a}_x^{\dagger}\hat{a}_x+i\hat{a}_x^{\dagger}\hat{a}_y-i\hat{a}_y^{\dagger}\hat{a}_x+\hat{a}_y^{\dagger}\hat{a}_y)$.
Hence notice that $\hat{N}_--\hat{N}_+=i(\hat{a}_x\otimes\hat{a}_y^{\dagger}-\hat{a}_x^{\dagger}\otimes\hat{a}_y)$, from which it directly follows that $\hat{L}_z=\hbar(\hat{N}_--\hat{N}_+)$. Then we have that $\hat{N}_-+\hat{N}_+=\hat{a}_x^{\dagger}\otimes \hat{a}_x+\hat{a}_y^{\dagger}\otimes \hat{a}_y$ which implies that $\hat{H}_{x,y}=\hbar\omega_{\perp}(\hat{N}_-+\hat{N}_++I)$.

\subsubsection*{g)}
First of all we study the action of $\hat{N}_{\pm}$ on $\hat{a}_{\mp}^{\dagger}\ket{n_{\pm}}$.  Now, if we use the commutation relations found in question $4.d$ we have:

$$\hat{N}_{+}\hat{a}_{-}^{\dagger}\ket{n_{+}}=\hat{a}_+^{\dagger}\hat{a}_+\hat{a}_-^{\dagger}\ket{n_+}=\hat{a}_-^{\dagger}\hat{a}_+^{\dagger}\hat{a}_+\ket{n_+}=n_+\hat{a}^{\dagger}_-\ket{n_+}$$
and 
$$\hat{N}_{-}\hat{a}_{+}^{\dagger}\ket{n_{-}}=\hat{a}_-^{\dagger}\hat{a}_-\hat{a}_+^{\dagger}\ket{n_-}=\hat{a}_+^{\dagger}\hat{a}_-^{\dagger}\hat{a}_-\ket{n_-}=n_-\hat{a}^{\dagger}_+\ket{n_-}$$
We already proved that we can co-diagonalize $\hat{N}_x$ and $\hat{N}_y$ in some common basis $\ket{\psi_{n_x,n_y}}$, such that
$\hat{N}_+\ket{\psi_{n_+,n_-}}=n_+\ket{\psi_{n_+,n_-}}$ and $\hat{N}_-\ket{\psi_{n_+,n_-}}=n_-\ket{\psi_{n_+,n_-}}$
Then we have that 
$$\ket{\psi_{n_++1,n_-}}=\frac{\hat{a}_+^{\dagger}\ket{\psi_{n_+,n_-}}}{||\hat{a}_+^{\dagger}\ket{\psi_{n_+,n_-}}||}=\frac{\hat{a}_+^{\dagger}\ket{\psi_{n_+,n_-}}}{\sqrt{n_++1}}$$

$$\ket{\psi_{n_+1,n_-+1}}=\frac{\hat{a}_-^{\dagger}\ket{\psi_{n_+,n_-}}}{||\hat{a}_-^{\dagger}\ket{\psi_{n_+,n_-}}||}=\frac{\hat{a}_-^{\dagger}\ket{\psi_{n_+,n_-}}}{\sqrt{n_-+1}}$$
since $||\hat{a}_+^{\dagger}\ket{\psi_{n_+,n_-}}||=\sqrt{\braket{\psi_{n_+,n_-}|a_+a_+^{\dagger}|\psi_{n_+,n_-}}}=\sqrt{\braket{\psi_{n_+,n_-}|a_+^{\dagger}a_++1|\psi_{n_+,n_-}}}=\sqrt{n_++1}$ and similarly for $\hat{a}_-$.
Therefore if we denote by $\ket{\psi_{0,0}}$ the fundamental state, for which $\hat{a}_+\ket{\psi_{0,0}}=0$ and $\hat{a}_-\ket{\psi_{0,0}}=0$ (which have to exist since otherwise we would get an infinite descent of eigenvalues of $n_+,n_-$, which is impossible for what we have already proved before, we get

$$\ket{\psi_{n_+,n_-}}=\frac{(\hat{a}_+^{\dagger})^{n_+}}{\sqrt{n_+!}}\frac{(\hat{a}_-^{\dagger})^{n_-}}{\sqrt{n_-!}}\ket{\psi_{0,0}}$$

Finally this is also the basis of diagonalization of $\hat{H}_{x,y}$ and $\hat{L}_z$ (since they decomposed as a sum/difference of $N_+,N_-$). Then we have that the eigenvalues of $\hat{H}_{x,y}$ are given by $\hbar\omega_{\perp}(n_-+n_++1)$ and the eigenvalues of $\hat{L}_z$ by $\hbar(n_--n_+)$.

\subsection*{5)}

\subsubsection*{a)}

If we denote by $\ket{0}$ the state for which $\hat{a}_+\ket{0}=0$ and $\hat{a}_-\ket{0}=0$ then we have that $\hat{H}_{x,y}\ket{0}=\hbar\omega_{\perp}$, i.e. $\ket{0}$ is the fundamental state for $\hat{H}_{x,y}$. Now we must have $(a_x+ia_y)\ket{0}=0$ and $(a_x-ia_y)\ket{0}=0$ from which we get that $a_x\ket{0}=0$ and $a_y\ket{0}=0$. Therefore it follows that 
\begin{align*}
\bigg(x+\frac{\hbar}{m\omega_{\perp}}\frac{d}{dx}\bigg)\psi_{0,0}(x,y)=0\\
\bigg(y+\frac{\hbar}{m\omega_{\perp}}\frac{d}{dy}\bigg)\psi_{0,0}(x,y)=0
\end{align*}

which implies that $\psi_{0,0}(x,y)=Ce^{-\frac{m\omega_{\perp}}{2\hbar}\rho^2}$, $C$ being the normalization constant:
$$C^2\int_{-\infty}^{\infty}\int_{-\infty}^{\infty} e^{-\frac{m\omega_{\perp}}{\hbar}(x^2+y^2)}=1\Rightarrow C=\sqrt{\frac{m\omega_{\perp}}{\pi \hbar}}$$

\subsubsection*{b)}

We have that $\psi_{1,0}(\rho,\theta)=\hat{a}_+^{\dagger}\psi_{0,0}(\rho)$. Now notice that
\begin{align*}
\hat{a}_{+}^{\dagger}&=\frac{1}{2}\bigg(\frac{\hat{x}}{a_0}-i\frac{\hat{p}_x}{p_0}-i\frac{\hat{y}}{a_0}-\frac{\hat{p}_y}{p_0}\bigg)=\frac{1}{2}\bigg(\frac{1}{a_0}(x-iy)-a_0\bigg(\frac{\p}{\p x}-i\frac{\p}{\p y}\bigg)\bigg)=\frac{1}{2}\bigg(\frac{1}{a_0}\rho e^{-i\theta}-a_0\bigg(\frac{\p}{\p (\rho\cos\theta)}-i\frac{\p}{\p (\rho\sin\theta)}\bigg)\bigg)\\
&=\frac{1}{2}\bigg(\frac{1}{a_0}\rho e^{-i\theta}-a_0\bigg(\cos\theta\frac{\p}{\p\rho}-\frac{1}{\rho}\sin\theta\frac{\p}{\p\theta}-i\sin\theta\frac{\p}{\p \rho}-i\frac{1}{\rho}\cos\theta\frac{\p}{\p\theta}\bigg)\bigg)\\
&=\frac{e^{-i\theta}}{2}\bigg(\frac{\rho}{a_0}-a_0\frac{\p}{\p\rho}+a_0\frac{i}{\rho}\frac{\p}{\p\theta}\bigg)
\end{align*}

and similarly $\hat{a}_-^{\dagger}=\frac{e^{i\theta}}{2}\big(\frac{\rho}{a_0}-a_0\frac{\p}{\p\rho}-a_0\frac{i}{\rho}\frac{\p}{\p\theta}\big)$.
Hence it follows that 
$$\psi_{1,0}(\rho,\theta)=\frac{e^{-i\theta}}{2}\bigg(\frac{\rho}{a_0}\psi_{0,0}(\rho)+a_0\frac{1}{a_0^2}\psi_{0,0}(\rho)\bigg)=\frac{\rho}{a_0^2\sqrt{\pi}}e^{-\frac{\rho^2}{2a_0^2}}e^{i\theta}$$
and similarly 
$$\psi_{0,1}(\rho,\theta)=\frac{\rho}{a_0^2\sqrt{\pi}}e^{-\frac{\rho^2}{2a_0^2}}e^{-i\theta}$$

\subsubsection*{c)}
We recall that the probability current is defined as $\vec{J}=-\frac{\hbar}{m}\text{Im}(\psi\nabla\psi^{*})$ and that we have the continuity equation $\nabla \cdot \vec{J}=-\frac{\p|\psi|^2}{\p t}$.

\subsubsection*{d)}

In this case we have that for $\ket{0}$
$$\mathbf{J}=0$$ since $\ket{0}$ is only real. Then for $\ket{\pm}$ we have:
$$\mathbf{J}_-=-\frac{\hbar}{m}\text{Im}\bigg[\frac{\rho}{a_0^2\sqrt{\pi}}e^{-\frac{\rho^2}{2a_0^2}}e^{-i\theta}\bigg(\frac{1}{a_0^2\sqrt{\pi}}e^{-\frac{\rho^2}{2a_0^2}}e^{i\theta}\mathbf{u}_{\rho}-\frac{\rho^2}{a_0^4\sqrt{\pi}}e^{-\frac{\rho^2}{2a_0^2}}e^{i\theta}\mathbf{u}_{\rho}+i\frac{1}{a_0^2\sqrt{\pi}}e^{-\frac{\rho^2}{2a_0^2}}e^{i\theta}\mathbf{u}_{\theta}\bigg)\bigg]=-\frac{\hbar}{m}\frac{\rho}{a_0^4\pi}e^{-\frac{\rho^2}{a_0^2}}\mathbf{u}_{\theta}$$
and similarly
$$\mathbf{J}_+=-\frac{\hbar}{m}\text{Im}\bigg[\frac{\rho}{a_0^2\sqrt{\pi}}e^{-\frac{\rho^2}{2a_0^2}}e^{i\theta}\bigg(\frac{1}{a_0^2\sqrt{\pi}}e^{-\frac{\rho^2}{2a_0^2}}e^{-i\theta}\mathbf{u}_{\rho}-\frac{\rho^2}{a_0^4\sqrt{\pi}}e^{-\frac{\rho^2}{2a_0^2}}e^{-i\theta}\mathbf{u}_{\rho}-i\frac{1}{a_0^2\sqrt{\pi}}e^{-\frac{\rho^2}{2a_0^2}}e^{-i\theta}\mathbf{u}_{\theta}\bigg)\bigg]=\frac{\hbar}{m}\frac{\rho}{a_0^4\pi}e^{-\frac{\rho^2}{a_0^2}}\mathbf{u}_{\theta}$$


\subsubsection*{e)}

We see that the probability currents, for $\ket{\pm}$, depend only on the position from the $z$ axis, $\rho$, and are oriented in planes that are perpendicular to the $z$ axis, i.e. along the $\mathbf{u}_{\theta}$ direction. Hence the distribution of particles will be forming a vortex that is null on the $z$ axis, highly concetrated in a zone around the $z$ axis and going to zero at infinity (which means far from the $z$ axis). This behaviour can be understood from the function $\rho e^{-\frac{\rho^2}{a_0^2}}$, which is zero at the center, increases up to a maximum and then goes to zero, giving the idea of a vortex.

\subsection*{6)}

\subsubsection*{a)}

Since it is an alkali metal, it has one electron in the outermost orbital. Then an isotope of the Rubidium will lack one neutron. Therefore the spin of the outer electron will cancel with the spin of the leftover proton in the nucleus, thus resulting in a total spin of zero (boson).

\subsubsection*{b)}
In the picture the vortex is not centered, in contrast to what we predicted before. 

\subsection*{7)}

\subsubsection*{a)}
we have that 
\begin{align*}
\psi(x,y)&=\alpha \sqrt{\frac{m\omega_{\perp}}{\pi \hbar}}e^{-\frac{m\omega_{\perp}}{2\hbar}\rho^2}+\beta\frac{\rho}{a_0^2\sqrt{\pi}}e^{-\frac{m\omega_{\perp}}{2\hbar}\rho^2}e^{i\theta}=0\Leftrightarrow 
\alpha+\beta\sqrt{\frac{m\omega_{\perp}}{\hbar}}\rho e^{i\theta}=0\\
&\Leftrightarrow \frac{\alpha}{\beta}=-\sqrt{\frac{m\omega_{\perp}}{\hbar}}\rho e^{i\theta}
\end{align*}
from which it follows that $\theta=\text{Arg}(\alpha/\beta)$ and $\rho^2=\frac{|\alpha|^2}{1-|\alpha|^2}\frac{\hbar}{m\omega_{\perp}}$.

\subsubsection*{b)}
Since the Hamiltonian is time independent, we can say that any wavefunction evolves according to the evolution operator $e^{-\frac{i}{\hbar}t\hat{H}}$. In particular any eigenstate of $\hat{H}_{x,y}$ will just gain a phase $e^{-\frac{i}{\hbar}tE_n}$. Hence we have
$$\ket{\psi(t)}=\alpha\ket{0}e^{-\frac{i}{\hbar}t\hbar\omega_{\perp}}+\beta\ket{+}e^{-\frac{i}{\hbar}t2\hbar \omega_{\perp}}=\alpha\ket{0}e^{-i\omega_{\perp}t}+\beta\ket{+}e^{-2i\omega_{\perp}t}$$
and from before it follows that 
$$\theta(t)=\text{Arg}\bigg(\frac{\alpha e^{i\omega_{\perp} t}}{\beta}\bigg)=\text{Arg}(\alpha/\beta)+\omega_{\perp}t$$


so we see that there is a precession of the axis of zero density, with precession frequency of $\omega_{\perp}$. 










\end{document}