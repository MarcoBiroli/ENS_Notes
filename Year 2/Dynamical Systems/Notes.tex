\documentclass[10pt,a4paper]{book}
\usepackage[utf8]{inputenc}
\usepackage[english]{babel}
\usepackage{amsmath}
\usepackage{amsfonts}
\usepackage{amssymb}
\usepackage{graphicx}
\usepackage{physics}
\usepackage{stmaryrd}
\usepackage{tikz}
\usepackage[left=2cm,right=2cm,top=2cm,bottom=2cm]{geometry}
\author{Marco Biroli}
\title{Dynamical behaviors in nonlinear and out of equilibrium systems}


\begin{document}
\maketitle
\tableofcontents

\chapter{Introduction}
\section{Definitions}
A dynamical system is simply a system of coupled ordinary differential equations. This type of problem occurs in many areas of physics and also outside of physics. The simplest example is analytical mechanics, whether we use Newton's, Lagrange's or Hamilton's Laws in the end the problem always comes down to solving a system of ordinary differential equations. In electromagnetism things become slightly more complicated. Instead of a set of ODE's we have a set of coupled PDE's, however after some simplifications we can reduce some problems to dynamical systems problems. 

\chapter{Characteristic features of nonlinear phenomena}
\section{Multiple solutions. Energy budget of the Earth.}
We take the energy balance of the earth as an example. The input in energy comes majorly from the power of the sun which we denote by $p_S$. The main output is from blackbody radiation in the infrared range to space which we can write as $\sigma T^4$. Hence we have for equation:
\[
C \dv{T}{t} = p_s(1 - \alpha(T)) - \sigma T^4
\]

\end{document}