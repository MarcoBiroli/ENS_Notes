\documentclass[10pt,a4paper]{article}
\usepackage[utf8]{inputenc}
\usepackage[english]{babel}
\usepackage{amsmath}
\usepackage{amsfonts}
\usepackage{amssymb}
\usepackage{graphicx}
\usepackage{physics}
\usepackage{stmaryrd}
\usepackage[left=2cm,right=2cm,top=2cm,bottom=2cm]{geometry}
\author{Marco Biroli}
\title{Midterm homework problems}

\begin{document}
\maketitle

\section{Divergence and Laplacian}

\begin{enumerate}

\item We have the definition of Christoffel symbols:
\[
\Gamma_{ij}^k = \pdv{\vb{e}_i}{x^j} \cdot \vb{e}^k
\]
Then we have that:
\[
\div \vb{V} = \partial_i ( V^j \vb{e_j})^i = \pdv{V^i}{x^i} + \Gamma^i_{i j} V^j = V^i_{, i} + \frac{1}{2} g^{im} ( g_{mi, j} + g_{mj, i} - g_{i j, m} ) V^j
\]
Then:
\[
...
\]

\item Since the determinant is an invariant scalar of the matrix then from the relation: $g^{\mu\nu} = g^{-1} c^{\mu \nu}$ we know that $c$ transforms in the exact same way as $g$ does. Since $g$ is a tensor then so is $c$. 

\item We have that:
\[
g = \sum_\nu g_{\mu \nu} c^{\mu \nu} \mbox{~hence~} \pdv{g}{g_{\mu \nu}} = \pdv{}{g_{\mu \nu}} \sum_{\nu'} g_{\mu \nu'} c^{\mu \nu'} = c^{\mu \nu}
\]

\item We have that:
\[
g^{\mu \nu} g_{\mu \nu, \gamma} = \partial_\gamma \log g 
\]
We have that:
\[
\partial_\gamma g (g_{\mu \nu}) = (\partial_\gamma g) g_{\mu \nu} + g \partial_\gamma g_{\mu \nu} 
\]
We have that:
\[
\partial_\gamma g = \pdv{}{g_{\mu \nu}} \pdv{g_{\mu \nu}}{\gamma} g = \pdv{}{g_{\mu \nu}} g_{\mu \nu, \gamma} g = g_{\mu \nu, \gamma} c^{\mu \nu}
\]
Hence:
\[
...
\]

\item We start from the end and we differentiate to obtain:
\[
\frac{1}{\sqrt{|g|}} \partial_\gamma (\sqrt{|g|} V^\gamma) = {V^\gamma}_{, \gamma} + \frac{1}{\sqrt{|g|}} V^\gamma \frac{1}{2 \sqrt{|g|}} \partial_\gamma |g| = {V^\mu}_{, \mu} + \frac{1}{2} V^\gamma \frac{\partial_\gamma |g|}{|g|} = {V^\mu}_{, \mu} + \frac{1}{2} V^\gamma \log|g|
\]
Now using question 4 we re-obtain the formula of question 1 and this concludes the proof.

\item Using the above formula by replacing: $V_\gamma = f_{, \gamma}$ (hence $V^\gamma = g^{\gamma \mu} f_{,\mu}$) we obtain:
\[
\laplacian f = \frac{1}{\sqrt{|g|}} \partial_\gamma( \sqrt{|g|} f^{,\gamma} ) = \frac{1}{\sqrt{|g|}} \partial_\gamma( \sqrt{|g|} g^{\gamma \mu}f_{,\mu} )
\] 

\item In spherical coordinates we have that:
\[
[g_{\mu \nu}] = \begin{pmatrix}
1 & 0 & 0\\
0 & r^2 & 0 \\
0 & 0 & r^2 \sin^2 \theta 
\end{pmatrix}
\]
Then in order to apply the previous formula we need to compute $g$ and $[g^{\mu \nu}]$. We have quite simply:
\[
g = r^4 \sin^2 \theta \mbox{~~and~~} g^{\mu \mu} = \frac{1}{g_{\mu \mu}} \mbox{~~and~~} g^{\mu \nu} = 0 \mbox{~~otherwise.}
\]
Plugging this in the previous formula we obtain:
\[
\laplacian f = \frac{1}{r^2 \sin \theta} \partial_\gamma(r^2 \sin\theta g^{\gamma \mu} f_{, \mu}) = \frac{1}{r^2 \sin \theta} \left( \partial_r (r^2 \sin \theta f_{,r}) + \partial_\theta (\sin \theta f_{, \theta}) + \partial_\varphi (\frac{1}{\sin \theta} f_{,\varphi}) \right)
\]
Now simplifying the derivatives gives:
\begin{align*}
\laplacian f &= \frac{1}{r^2 \sin \theta} \left( \sin \theta \partial_r (r^2 f_{,r}) + \partial_{\theta} (\sin \theta f_{, \theta}) + \frac{1}{\sin \theta} \partial_\varphi f_{,\varphi}  \right) \\
&= \frac{1}{r^2} \partial_r (r^2 f_{,r}) + \frac{1}{r^2 \sin \theta} \partial_\theta (\sin \theta f_{, \theta}) + \frac{1}{r^2 \sin^2 \theta}\partial_\varphi f_{,\varphi}
\end{align*}

\item Repeating an identical argument but using:
\[
[g_{\mu \nu}] = \begin{pmatrix}
1 & 0 & 0\\
0 & 1 & 0\\
0 & 0 & r^2
\end{pmatrix}
\]
Gives us immediately that:
\[
\laplacian f = \frac{1}{r} \partial_\gamma (r g^{\gamma \mu}f_{, \mu}) = r^{-1} ( \partial_z (r f_{,z} + \partial_{r} (r f_{,r}) + \partial_\phi (r^{-1} f_{,\phi})  ) = f_{,zz} + r^{-1} f_{,r} + f_{,rr} + r^{-2} f_{,\phi\phi}
\]
\end{enumerate}

\section{Rotating coordinate frame.}

\begin{enumerate}

\item We have that:
\[
t = t \mbox{~~and~~} z = z' \mbox{~~and~~} r = r' \mbox{~~and~~} \phi = \phi' - \Omega t
\]
Hence we immediately get that:
\[
\dd t = \dd t \mbox{~~and~~} \dd z = \dd z' \mbox{~~and~~} \dd r = \dd r' \mbox{~~and~~} \dd \phi = \dd \phi' - t \dd \Omega  - \Omega \dd t = \dd \phi' - \Omega \dd t
\]
Where in the last equality we add the assumption that we place ourselves in a rotating frame at constant angular velocity. Now plugging this in the expression for a line element we obtain:
\begin{align*}
\dd s^2 &= -c^2 \dd t^2 + (\dd z')^2 + (\dd r')^2 + (r')^2 (\dd \phi')^2 = - c^2 \dd t^2 + \dd z^2 + \dd r^2 + r^2 (\dd \phi + \Omega \dd t)^2\\
&= - c^2 \dd t^2 + \dd z^2 + \dd r^2 + r^2 \dd \phi^2 + r^2 \Omega^2 \dd t^2 + 2 r^2 \Omega \dd \phi \dd t\\
&= (r^2 \Omega^2 - c^2) \dd t^2 + \dd z^2 + \dd r^2 + r^2 \dd \phi^2 + 2r^2 \Omega \dd t \dd \phi
\end{align*}
Hence we also get:
\[
[g_{\mu \nu}] = \begin{pmatrix}
(r^2 \Omega^2 - c^2) & 0 & 0 & r^2 \Omega\\
0 & 1 & 0 & 0\\
0 & 0 & 1 & 0\\
r^2 \Omega & 0 & 0 & r^2
\end{pmatrix}
\]

\item The inverse can be immediately obtained through it's cofactor formulation and gives:
\[
[g^{\mu\nu}] = \left(
\begin{array}{cccc}
 -\frac{1}{c^2} & 0 & 0 & \frac{\Omega }{c^2} \\
 0 & 1 & 0 & 0 \\
 0 & 0 & 1 & 0 \\
 \frac{\Omega }{c^2} & 0 & 0 & \frac{c^2 - r^2 \Omega ^2}{c^2 r^2} \\
\end{array}
\right) \mbox{~~and~~} g = -c^2 r^2 
\]

\item We have that:
\[
\begin{pmatrix}
t'\\
x'\\
y'\\
z'
\end{pmatrix} = \begin{pmatrix}
1 & 0 & 0 & 0\\
0 & \cos \Omega t & \sin \Omega t & 0\\
0 & - \sin \Omega t & \cos \Omega t & 0\\
0 & 0 & 0 & 1
\end{pmatrix} \begin{pmatrix}
t\\
x\\
y\\
z
\end{pmatrix}
\] 
Now notice that the transition matrix is orthogonal hence we immediately have that:
\[
\begin{pmatrix}
1 & 0 & 0 & 0\\
0 & \cos \Omega t & -\sin \Omega t & 0\\
0 & \sin \Omega t & \cos \Omega t & 0\\
0 & 0 & 0 & 1
\end{pmatrix}
\begin{pmatrix}
t'\\
x'\\
y'\\
z'
\end{pmatrix}
= \begin{pmatrix}
t\\
x\\
y\\
z
\end{pmatrix}
\]
Hence we obtain immediately that:
\begin{align*}
\dd s^2 &= - c^2 \dd t^2 + \dd x^2 + \dd y^2 + \dd z^2 = -c^2 \dd t^2 + (\dd (x' \cos \Omega t - y' \sin \Omega t))^2 + (\dd (x' \sin \Omega t + y' \cos \Omega t))^2 + (\dd z')^2\\
&= - c^2 \dd t^2 + (\cos\Omega t \dd x' - x' \Omega \dd t \sin\Omega t  - \sin \Omega t \dd y' - y' \Omega \dd t \cos \Omega t)^2 = ...
\end{align*}

\item We have that:
\[
\begin{pmatrix}
-1 + h_{00} & h_{01} & h_{02} & h_{03}\\
h_{10} & 1 + h_{11} & h_{12} & h_{13}\\
h_{2 0} & h_{21} & 1 + h_{22} & h_{23}\\
h_{30} & h_{31} & h_{32} & 1 + h_{33}
\end{pmatrix}
= 
\begin{pmatrix}
-(1 - (x^2 + y^2) \Omega^2) & \Omega y & -\Omega x & 0\\
\Omega y & 1 & 0 & 0\\
-\Omega x & 0 & 1 & 0\\
0 & 0 & 0 & 1
\end{pmatrix}
\]
Hence we get:
\[
[h_{\mu \nu}] = \begin{pmatrix}
(x^2 + y^2) \Omega^2 & \Omega y & - \Omega x & 0\\
\Omega y & 0 & 0 & 0\\
-\Omega x & 0 & 0 & 0\\
0 & 0 & 0 & 0
\end{pmatrix}
\]

\end{enumerate}

\end{document}