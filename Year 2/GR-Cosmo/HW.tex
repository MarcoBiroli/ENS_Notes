\documentclass[10pt,a4paper]{article}
\usepackage[utf8]{inputenc}
\usepackage[english]{babel}
\usepackage{amsmath}
\usepackage{amsfonts}
\usepackage{amssymb}
\usepackage{graphicx}
\usepackage{physics}
\usepackage{stmaryrd}
\usepackage[left=2cm,right=2cm,top=2cm,bottom=2cm]{geometry}
\author{Marco Biroli}
\title{Midterm homework problems}

\begin{document}
\maketitle

\section{Divergence and Laplacian}

\begin{enumerate}

\item We have the definition of Christoffel symbols:
\[
\Gamma_{ij}^k = \pdv{\vb{e}_i}{x^j} \cdot \vb{e}^k
\]
Then we have that:
\[
\div \vb{V} = \partial_i ( V^j \vb{e_j})^i = \pdv{V^i}{x^i} + \Gamma^i_{i j} V^j = V^i_{, i} + \frac{1}{2} g^{im} ( g_{mi, j} + g_{mj, i} - g_{i j, m} ) V^j
\]
Then:
\[
...
\]

\item Since the determinant is an invariant scalar of the matrix then from the relation: $g^{\mu\nu} = g^{-1} c^{\mu \nu}$ we know that $c$ transforms in the exact same way as $g$ does. Since $g$ is a tensor then so is $c$. 

\item We have that:
\[
g = \sum_\nu g_{\mu \nu} c^{\mu \nu} \mbox{~hence~} \pdv{g}{g_{\mu \nu}} = \pdv{}{g_{\mu \nu}} \sum_{\nu'} g_{\mu \nu'} c^{\mu \nu'} = c^{\mu \nu}
\]

\item We have that:
\[
g^{\mu \nu} g_{\mu \nu, \gamma} = \partial_\gamma \log g 
\]
We have that:
\[
\partial_\gamma g (g_{\mu \nu}) = (\partial_\gamma g) g_{\mu \nu} + g \partial_\gamma g_{\mu \nu} 
\]
We have that:
\[
\partial_\gamma g = \pdv{}{g_{\mu \nu}} \pdv{g_{\mu \nu}}{\gamma} g = \pdv{}{g_{\mu \nu}} g_{\mu \nu, \gamma} g = g_{\mu \nu, \gamma} c^{\mu \nu}
\]
Hence:
\[
...
\]

\item We start from the end and we differentiate to obtain:
\[
\frac{1}{\sqrt{|g|}} \partial_\gamma (\sqrt{|g|} V^\gamma) = {V^\gamma}_{, \gamma} + \frac{1}{\sqrt{|g|}} V^\gamma \frac{1}{2 \sqrt{|g|}} \partial_\gamma |g| = {V^\mu}_{, \mu} + \frac{1}{2} V^\gamma \frac{\partial_\gamma |g|}{|g|} = {V^\mu}_{, \mu} + \frac{1}{2} V^\gamma \log|g|
\]
Now using question 4 we re-obtain the formula of question 1 and this concludes the proof.

\item Using the above formula by replacing: $V_\gamma = f_{, \gamma}$ (hence $V^\gamma = g^{\gamma \mu} f_{,\mu}$) we obtain:
\[
\laplacian f = \frac{1}{\sqrt{|g|}} \partial_\gamma( \sqrt{|g|} f^{,\gamma} ) = \frac{1}{\sqrt{|g|}} \partial_\gamma( \sqrt{|g|} g^{\gamma \mu}f_{,\mu} )
\] 

\item In spherical coordinates we have that:
\[
[g_{\mu \nu}] = \begin{pmatrix}
1 & 0 & 0\\
0 & r^2 & 0 \\
0 & 0 & r^2 \sin^2 \theta 
\end{pmatrix}
\]
Then in order to apply the previous formula we need to compute $g$ and $[g^{\mu \nu}]$. We have quite simply:
\[
g = r^4 \sin^2 \theta \mbox{~~and~~} g^{\mu \mu} = \frac{1}{g_{\mu \mu}} \mbox{~~and~~} g^{\mu \nu} = 0 \mbox{~~otherwise.}
\]
Plugging this in the previous formula we obtain:
\[
\laplacian f = \frac{1}{r^2 \sin \theta} \partial_\gamma(r^2 \sin\theta g^{\gamma \mu} f_{, \mu}) = \frac{1}{r^2 \sin \theta} \left( \partial_r (r^2 \sin \theta f_{,r}) + \partial_\theta (\sin \theta f_{, \theta}) + \partial_\varphi (\frac{1}{\sin \theta} f_{,\varphi}) \right)
\]
Now simplifying the derivatives gives:
\begin{align*}
\laplacian f &= \frac{1}{r^2 \sin \theta} \left( \sin \theta \partial_r (r^2 f_{,r}) + \partial_{\theta} (\sin \theta f_{, \theta}) + \frac{1}{\sin \theta} \partial_\varphi f_{,\varphi}  \right) \\
&= \frac{1}{r^2} \partial_r (r^2 f_{,r}) + \frac{1}{r^2 \sin \theta} \partial_\theta (\sin \theta f_{, \theta}) + \frac{1}{r^2 \sin^2 \theta}\partial_\varphi f_{,\varphi}
\end{align*}

\item Repeating an identical argument but using:
\[
[g_{\mu \nu}] = \begin{pmatrix}
1 & 0 & 0\\
0 & 1 & 0\\
0 & 0 & r^2
\end{pmatrix}
\]
Gives us immediately that:
\[
\laplacian f = \frac{1}{r} \partial_\gamma (r g^{\gamma \mu}f_{, \mu}) = r^{-1} ( \partial_z (r f_{,z} + \partial_{r} (r f_{,r}) + \partial_\phi (r^{-1} f_{,\phi})  ) = f_{,zz} + r^{-1} f_{,r} + f_{,rr} + r^{-2} f_{,\phi\phi}
\]
\end{enumerate}

\section{Rotating coordinate frame.}

\begin{enumerate}

\item We have that:
\[
t = t \mbox{~~and~~} z = z' \mbox{~~and~~} r = r' \mbox{~~and~~} \phi = \phi' - \Omega t
\]
Hence we immediately get that:
\[
\dd t = \dd t \mbox{~~and~~} \dd z = \dd z' \mbox{~~and~~} \dd r = \dd r' \mbox{~~and~~} \dd \phi = \dd \phi' - t \dd \Omega  - \Omega \dd t = \dd \phi' - \Omega \dd t
\]
Where in the last equality we add the assumption that we place ourselves in a rotating frame at constant angular velocity. Now plugging this in the expression for a line element we obtain:
\begin{align*}
\dd s^2 &= -c^2 \dd t^2 + (\dd z')^2 + (\dd r')^2 + (r')^2 (\dd \phi')^2 = - c^2 \dd t^2 + \dd z^2 + \dd r^2 + r^2 (\dd \phi + \Omega \dd t)^2\\
&= - c^2 \dd t^2 + \dd z^2 + \dd r^2 + r^2 \dd \phi^2 + r^2 \Omega^2 \dd t^2 + 2 r^2 \Omega \dd \phi \dd t\\
&= (r^2 \Omega^2 - c^2) \dd t^2 + \dd z^2 + \dd r^2 + r^2 \dd \phi^2 + 2r^2 \Omega \dd t \dd \phi
\end{align*}
Hence we also get:
\[
[g_{\mu \nu}] = \begin{pmatrix}
(r^2 \Omega^2 - c^2) & 0 & 0 & r^2 \Omega\\
0 & 1 & 0 & 0\\
0 & 0 & 1 & 0\\
r^2 \Omega & 0 & 0 & r^2
\end{pmatrix}
\]

\item The inverse can be immediately obtained through it's cofactor formulation and gives:
\[
[g^{\mu\nu}] = \left(
\begin{array}{cccc}
 -\frac{1}{c^2} & 0 & 0 & \frac{\Omega }{c^2} \\
 0 & 1 & 0 & 0 \\
 0 & 0 & 1 & 0 \\
 \frac{\Omega }{c^2} & 0 & 0 & \frac{c^2 - r^2 \Omega ^2}{c^2 r^2} \\
\end{array}
\right) \mbox{~~and~~} g = -c^2 r^2 
\]

\item We have that:
\[
\begin{pmatrix}
t'\\
x'\\
y'\\
z'
\end{pmatrix} = \begin{pmatrix}
1 & 0 & 0 & 0\\
0 & \cos \Omega t & \sin \Omega t & 0\\
0 & - \sin \Omega t & \cos \Omega t & 0\\
0 & 0 & 0 & 1
\end{pmatrix} \begin{pmatrix}
t\\
x\\
y\\
z
\end{pmatrix}
\] 
Now notice that the transition matrix is orthogonal hence we immediately have that:
\[
\begin{pmatrix}
1 & 0 & 0 & 0\\
0 & \cos \Omega t & -\sin \Omega t & 0\\
0 & \sin \Omega t & \cos \Omega t & 0\\
0 & 0 & 0 & 1
\end{pmatrix}
\begin{pmatrix}
t'\\
x'\\
y'\\
z'
\end{pmatrix}
= \begin{pmatrix}
t\\
x\\
y\\
z
\end{pmatrix}
\]
Hence we obtain immediately that:
\begin{align*}
\dd s^2 &= - c^2 \dd t^2 + \dd x^2 + \dd y^2 + \dd z^2 = -c^2 \dd t^2 + (\dd (x' \cos \Omega t - y' \sin \Omega t))^2 + (\dd (x' \sin \Omega t + y' \cos \Omega t))^2 + (\dd z')^2\\
&= - c^2 \dd t^2 + (\cos\Omega t \dd x' - x' \Omega \dd t \sin\Omega t  - \sin \Omega t \dd y' - y' \Omega \dd t \cos \Omega t)^2 = ...
\end{align*}

\item We have that:
\[
\begin{pmatrix}
-1 + h_{00} & h_{01} & h_{02} & h_{03}\\
h_{10} & 1 + h_{11} & h_{12} & h_{13}\\
h_{2 0} & h_{21} & 1 + h_{22} & h_{23}\\
h_{30} & h_{31} & h_{32} & 1 + h_{33}
\end{pmatrix}
= 
\begin{pmatrix}
-(1 - (x^2 + y^2) \Omega^2) & \Omega y & -\Omega x & 0\\
\Omega y & 1 & 0 & 0\\
-\Omega x & 0 & 1 & 0\\
0 & 0 & 0 & 1
\end{pmatrix}
\]
Hence we get:
\[
[h_{\mu \nu}] = \begin{pmatrix}
(x^2 + y^2) \Omega^2 & \Omega y & - \Omega x & 0\\
\Omega y & 0 & 0 & 0\\
-\Omega x & 0 & 0 & 0\\
0 & 0 & 0 & 0
\end{pmatrix}
\]

\end{enumerate}

\section{Frame dragging by a moving rod.}

\begin{enumerate}

\item We have the equation:
\[
\laplacian \Phi = 4\pi (G_N/c^2)\rho 
\]
From symmetry arguments we know already that $\Phi$ will only be a function of $r$. Hence we have that by plugging the expression of the laplacian found in part 1 we obtain:
\[
\Phi_{,zz} + \Phi_{,rr} + \Phi_{,r}/r + \Phi_{,\phi\phi}/r^2 = \Phi_{,rr} + \Phi_{,r}/r =  4 \pi (G_N/c^2) \rho
\]
This can be re-written as:
\[
\partial_r(r \Phi_{,r}) = 4 \pi (G_N /c^2)\rho r \Rightarrow r \Phi_{,r} = 4\pi (G_N/c^2) \rho r^{2}/2 + c
\]
Which gives immediately through integration a solution of the form:
\[
\Phi(r) = \pi (G_N/c^2) \rho r^2  + c \log(r) + c'
\]
Now the condition $\Phi_{,r}(0) = 0$ ensures that $c = 0$ and the condition $\Phi(R) = 0$ ensures that $c' = - \pi (G_N/c^2) \rho R^2$ hence the final solution is given by:
\[
\Phi(r) = \pi(G_N/c^2)\rho(r^2 - R^2)
\]

\item We have that:
\[
\overline{h_{\mu \nu}} = -4 \Phi \delta^0_\mu \delta^0_\nu
\]
Hence we get that:
\[
h_{\mu \nu} - \frac{1}{2} \eta_{\mu \nu} h = - 4 \Phi \delta^0_\mu \delta^0_\nu
\]
Now we have that:
\[
\eta_{\mu\nu}(\overline{h}_{\mu\nu}) = h - \frac{1}{2} Tr(\eta) h = h - 2h = - h
\]
Which when plugged in the previous equation gives immediately that:
\[
-h = 4 \Phi \Leftrightarrow h = -4 \Phi
\]
Hence we obtain that:
\[
h_{\mu \nu}(r) = - 2 \Phi \delta_{\mu \nu}
\]


\item We have that:
\[
\dd s^2 = (1 - 2 \Phi) (\dd r^2 + r^2 \dd \phi^2)
\]
Now we apply the following transformations:
\[
r' = (1 + r \Phi_{,r})(1 - \Phi) r  \mbox{~~and~~} \phi' = (1 - \Phi_{,r}/r)\phi
\]
Using our previous results we can re-write this as:
\[
r' = (1 + 2\alpha r^2)(1 - \alpha(r^2 - R^2)) r \mbox{~~and~~} \phi' = (1 - 2\alpha) \phi 
\]
Hence we have that:
\begin{align*}
\dd r' &= \dd (r \Phi_{,r}) (1 - \Phi) r + (1 + r \Phi_{,r}) \dd(-\Phi) r + (1 + r \Phi_{,r})(1 - \Phi)\dd r\\
&= 4\pi(G_N/c^2)\rho r \dd r (1 - \Phi) r - (1 + r \Phi_{,r}) \Phi_{,r} \dd r + (1 + r \Phi_{,r})(1 - \Phi)\dd r \\
&= \dd r \left( \alpha r - \alpha r \Phi - \Phi_{,r} - r \Phi_{,r}^2 + 1 - \Phi + r \Phi_{,r} - r \Phi_{,r}^2 \right)
\end{align*}
Similarly we get:
\[
\dd \phi' = (1 - \Phi_{,r}/r)\dd \phi - (\dd \Phi_{,r})/r \Phi + \Phi_{,r}/r^2 \phi \dd r = (1 - \Phi_{,r}/r)\dd \phi - ()/r \Phi + \Phi_{,r}/r^2 \phi \dd r
\]

\item 

\item Hence we obtain:
\[
[h_{\alpha' \beta'}] = 
-2 \Phi(r')
\begin{pmatrix}
1 & & -2v/c\\
 & \vb{I} & \\
 -2v/c & & 1
\end{pmatrix}
\]

\item \begin{enumerate}

\item We use the geodesic equation:
\[
\dv[2]{z}{\tau} = \Gamma^z_{\mu \nu} \dv{x^\mu}{\tau} \dv{x^\nu}{\tau}
\]
Then:
\[
\Gamma^\nu_{\mu \lambda} = \frac{1}{2}\eta^{\nu \gamma}(\partial_\mu h_{\gamma \lambda} + \partial_{\lambda} h_{\gamma \mu} - \partial_\gamma h_{\lambda \mu})
\]
We start by computing the Christoffel symbols:
\begin{align*}
\Gamma^z_{00} &= \frac{1}{2} \eta^{z \gamma}( \partial_0 h_{\gamma 0} + \partial_0 h_{\gamma 0} - \partial_\gamma h_{00}) = \partial_0 h_{z 0} - \frac{1}{2} \partial_z h_{0 0} = \partial_0 4 \Phi(r) v /c + \partial_z \Phi(r) = 0\\
\Gamma^z_{0 1} &= \frac{1}{2} \eta^{z \gamma} (\partial_0 h_{\gamma 1} + \partial_1 h_{\gamma 0} - \partial_\gamma h_{10}) = \frac{1}{2}(\partial_0 h_{z 1} + \partial_1 h_{z 0} - \partial_z h_{10}) = \frac{1}{2}(0 + \partial_x 4 \Phi(r) v /c + 0) = 2 v /c\,  \partial_x \Phi(r)\\
\Gamma^z_{02} &= \frac{1}{2}(\partial_0 h_{z2} + \partial_2 h_{z0} - \partial_z h_{20}) = 2v/c\, \partial_y \Phi(r)\\
\Gamma^z_{0 3} &= \frac{1}{2}(\partial_0 h_{z3} + \partial_3 h_{z3} - \partial_z h_{30} ) = \frac{1}{2}(-2\partial_z \Phi(r) - 4v/c \partial_z \Phi(r) ) = -(1 - 2v/c) \partial_z \Phi(r) = 0\\
\Gamma^z_{11} &= \frac{1}{2} (\partial_1 h_{z 1} + \partial_1 h_{z 1} - \partial_z h_{11}) =  \partial_z \Phi(r) = 0 \\
\Gamma^z_{12} &= \frac{1}{2}(\partial_1 h_{z 2} + \partial_2 h_{z 1} - \partial_z h_{21}) = 0\\
\Gamma^z_{13} &= \frac{1}{2}( \partial_1 h_{z3} + \partial_3 h_{z1} - \partial_z h_{31} ) = - \partial_x \Phi(r)\\
\Gamma^{z}_{22} &= \frac{1}{2}(\partial_2 h_{z2} + \partial_2 h_{z2} - \partial_z h_{22}) = \partial_z \Phi(r) = 0\\
\Gamma^{z}_{23} &= \frac{1}{2}(\partial_2 h_{z 3} + \partial_3 h_{z2} - \partial_z h_{32}) = - \partial_y \Phi(r)\\
\Gamma^z_{33} &= \frac{1}{2} (\partial_{3} h_{z3} + \partial_3 h_{z3} - \partial_z h_{33}) = 0
\end{align*}
Hence the geodesic equation becomes:
\[
\ddot{z} = \Gamma^z_{01} \dv{t}{\tau} \dv{x}{\tau} + \Gamma^z_{02} \dv{t}{\tau} \dv{y}{\tau} + \Gamma^z_{13} \dv{x}{\tau}\dv{z}{\tau} + \Gamma^z_{23} \dv{y}{\tau}\dv{z}{\tau}
\]
Plugging in the values we get:
\[
\ddot{z} = 2v/c( \partial_x \Phi(r) \gamma \dot{x} + \partial_y \Phi(r) \gamma \dot{y}) - \dot{z}( \dot{x} \partial_x \Phi(r)  + \dot{y}\partial_y \Phi(r)  )  = (\dot{\vb{x}} \cdot \grad \Phi) (\frac{2v}{c} \gamma - \dot{z})
\]

\end{enumerate}


\end{enumerate}

\section{Frame-dragging inside a rotating cylinder.}

\begin{enumerate}

\item From symmetry arguments we know that $\Phi$ is only a function of $x$ and $y$. Hence we have that:
\[
\grad \Phi_{\mu \nu}(\vb{x}) = \kappa \int_{-h}^{h} \oint_C \dd z' \dd x' \dd y' T_{\mu \nu}(\vb{x}') (\vb{x} - \vb{x'})/|\vb{x} - \vb{x'}|^3 
\]
Now we use the well-know distribution formula:
\[
\div \left( \frac{\vb{r}}{|\vb{r}|^3} \right) = \delta(\vb{r})
\]
Which immediately gives us that:
\[
\laplacian \Phi_{\mu \nu}(\vb{x}) = \int T_{\mu \nu}(\vb{x'}) \delta(\vb{x} - \vb{x'}) \dd \vb{x'} = T_{\mu \nu}(\vb{x})
\]
This is the differential form of Gauss's Law for $\grad\Phi$. This also tells us that:
\[
H 2 \pi r \grad \Phi_{\mu \nu}(r) = \oint_{\partial V} \grad \Phi_{\mu \nu} \cdot \dd \vb{A} = \iiint_V \laplacian \Phi_{\mu \nu} \dd V = \iiint_V T_{\mu \nu}(\vb{x}) \dd V = T
\]
Hence in order to write this as similar to Gauss's Law for electromagnetism we can re-write this as:
\[
\grad \Phi_{\mu \nu}(r) = \frac{T}{2\pi r H}
\]
Hence we obtain that:
\[
\grad \Phi = \frac{\rho c^2}{2\pi r H} \begin{pmatrix}
\pi r^2 H & 0 & \pi r^2 H \Omega & 0\\
0 & 0 & 0 & 0\\
\pi r^2 H \Omega & 0 & 0 & 0\\
0 & 0 & 0 & 0
\end{pmatrix} = \frac{\rho r c^2}{2} \begin{pmatrix}
1 & 0 & \Omega & 0\\
0 & 0 & 0 & 0\\
\Omega & 0 & 0 & 0\\
0 & 0 & 0 & 0
\end{pmatrix}
\]

\item 

\end{enumerate}

\end{document}