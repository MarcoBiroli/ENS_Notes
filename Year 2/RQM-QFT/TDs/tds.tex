\documentclass[10pt,a4paper]{book}
\usepackage[utf8]{inputenc}
\usepackage[english]{babel}
\usepackage{amsmath}
\usepackage{amsfonts}
\usepackage{amssymb}
\usepackage{graphicx}
\usepackage{physics}
\usepackage{stmaryrd}
\usepackage[left=2cm,right=2cm,top=2cm,bottom=2cm]{geometry}
\author{Marco Biroli}
\title{TDs - QFT}


\begin{document}
\maketitle
\chapter{TD1}
\section{Matrix Groups}
\section{The relationship between $SO(3)$ and $SU(2)$.}
\section{Representations of $SU(2)$.}

\begin{enumerate}

\item An immediate computation yields the desired result

\item Let $\ket{a}$ an eigenvector of $\vu{J}^2$ then:
\[
\vu{J}^2 \ket{a} = a \ket{a} \Rightarrow \bra{a} \vu{J}^2 \ket{a} = a \braket{a} \Rightarrow || \vu{J} \ket{a} ||^2 = a || \ket{a} || \Rightarrow a > 0
\]
We propose as a writing for them $j(j+1)$ notice that:
\[
j(j+1) = x \Leftrightarrow j^2 + j - x = 0 \Rightarrow j = \frac{-j + \sqrt{j^2 + 4 x}}{2}
\]
Hence the writing as $j(j+1)$ is not restrictive and covers all of $\mathbb{R}^+$.

\item Let $\ket{v}$ an eigenvector of $\vu{J}^2$ and $\vu{J_3}$ with eigenvalues $j(j+1)$ and $m$. Then:
\[
\vu{J}^2 \vu{J_+} \ket{v} = \vu{J_+}\vu{J}^2 \ket{v} = j (j + 1) \vu{J_+} \ket{v}
\]
Since the operator $\vu{J}^2$ commutes with the $\vu{J_i}$. Then:
\[
\vu{J_3} \vu{J_+} \ket{v} = (\vu{J_+} \vu{J_3} + [\vu{J_3}, \vu{J_+}] )\ket{v} = (m \vu{J_+} + i\vu{J_2} + 1 \vu{J_1}) \ket{v} = ( m + 1 ) \vu{J_+} \ket{v}  
\]
Identically for $\vu{J_-}$ we obtain the same thing but with $m - 1$ as the eigenvalue for $\vu{J_3}$.

\item Assume that there is no such vector than the ladder operator would span an infinite family of eigenvectors of $\vu{J_3}$ and $\vu{J_+}$ and hence $V$ would be infinite dimensional.

\item We have that:
\[
\vu{J_-}\vu{J_+} = \vu{J_1}^2 - i [\vu{J_1}, \vu{J_2}] + \vu{J_2}^2 = \vu{J}^2 - \vu{J_3}^2 + \vu{J_3}
\]
Then applying this for $\ket{v_0}$ we get:
\[
\vu{J_-}\vu{J_+}\ket{v_0} = 0 = (j(j + 1) - m_0^2 + m_0) \ket{v_0} \Rightarrow j(j+1) = m_0(m_0 + 1) 
\]

\item An identical argument tells us that successive application of the lowering ladder operator must lead to a vanishing state. Then from definition we have that:
\[
\ket{w_0} = (\vu{J_-})^k \ket{v_0} \Rightarrow m'_0 = m_0 - k
\]

\item Similarly as before we get the exact same result but with a minus sign.

\item We then have the system:
\[
\begin{cases}
j(j+1) = m_0 (m_0 + 1)\\
j(j + 1) = (m_0 - k) (m_0 - k - 1)
\end{cases}
\Rightarrow
\begin{cases}
j(j+1) = m_0(m_0 + 1)\\
k^2 + k = 2 m_0( 1 +  k)
\end{cases}
\Rightarrow
\begin{cases}
j = \frac{k}{2}\\
\frac{k}{2} = m_0
\end{cases}
\]

\item We have that $\vu{J_+}$ sends $\ket{j, m}$ to $\ket{j, m+1}$ and similarly $\vu{J_-}$ sends $\ket{j, m}$ to $\ket{j, m - 1}$. Then we get that:
\[
\vu{J_+} \ket{j, m} = x \ket{j, m + 1} \Rightarrow \bra{j, m} \vu{J_-} \vu{J_+} \ket{j, m} = |x|^2 = j(j+1) - m(m+1)
\]
Hence we obtain:
\[
x = \sqrt{j(j+1) - m(m+1)}
\]
Then we have that:
\[
\vu{J_1} \ket{j, m} = \frac{\vu{J_+} + \vu{J_-}}{2} \ket{j, m} = \frac{x}{2} \left( \ket{j, m+1} + \ket{j, m-1} \right)
\]
Similarly:
\[
\vu{J_2} \ket{j, m} = \frac{\vu{J_+} - \vu{J_-}}{2i} \ket{j, m} = \frac{x}{2i} \left( \ket{j, m+1} - \ket{j, m-1} \right)
\]

\item Since $\vu{J}^2$ commutes with the $\vu{J_i}$ we know that the eigenspaces of $\vu{J}^2$ are sub-representations of $SU(2)$. We now restrict ourselves to one eigenspace, call it $\tilde{V}_j$ corresponding to the eigenvalue $j(j+1)$. As said previously there must be at least one eigenvector of $\vu{J}^2$ and $\vu{J_3}$ which is killed by $\vu{J_+}$ call it $\ket{j, j, 1}$. Then from this eigenvector we can build $\ket{j, m, 1} = \vu{J_-}^{j - m} \ket{j, j, 1}$. Which is an irreducible subspace of $\tilde{V}_j$. Then we can write $\tilde{V}_j = V_j^1 \oplus \tilde{V}_j'$. We can then repeat the process on $\tilde{V}_j'$ until we spanned the whole space. Then we have:
\[
V = V_0^1 \oplus \cdots \oplus V^{n_0}_0 \oplus V_{1/2}^1 \oplus \cdots \oplus V_{1/2}^{n_{1/2}} \oplus \cdots 
\]

\item We have that $\vec{L} = \vec{R} \land \vec{P}$ where $\vec{R}$ and $\vec{P}$ are operators on $L^2(\mathbb{R}^3)$ where $[R_j, P_k] = i \delta_{jk}$. Then we have that $[L_a, L_b] = i\varepsilon_{a b c} L_c$. Then the space we describe is $V : \{ \psi : S^2 \to \mathbb{C}\}$ and the spherical harmonic decomposition tells us that:
\[
\psi(\theta, \varphi) = \sum_{\ell = 0}^{+\infty}\sum_{m = -\ell}^\ell a_{\ell,m} Y_\ell^m(\theta, \varphi)
\]
Furthermore we have that:
\[
\vec{L}^2 = Y_\ell^m = \ell(\ell+1) Y_{\ell}^m \mbox{~~and~~} L_3 Y_\ell^m = m Y_\ell^m
\]
Hence the subspace $V_\ell = \text{Span}(Y_\ell^{-\ell}, \cdots, Y_\ell^\ell)$ is stable under rotation and $V = V_0 \oplus V_1 \oplus V_2 \oplus \cdots $

\item We have:
\[
e^{2i\pi \vu{J_3}} \ket{j, m} = e^{2 i \pi m} \ket{j, m}
\]
Now if $j$ is an integer we have that $m \in \mathbb{Z}$ and hence $e^{2 i \pi \vu{J_3}} = \text{Id}$. However if $j$ is a half integer then $m$ is also a half integer and hence $e^{2 i \pi \vu{J_3}} = -\text{Id}$.

\item In QM for example we usually consider the wavefunctions of one particle with no spin we will use the space $L^2(\mathbb{R}^3, \mathbb{C})$ however now if we introduce spin we will consider $L^2(\mathbb{R}^3, \mathbb{C}) \otimes \mathbb{C}^2$ or similarly if we consider two particles we need to consider $L^2(\mathbb{R}^3, \mathbb{C}) \otimes L^2(\mathbb{R}^3, \mathbb{C})$. Then we know also that:
\[
V_{j_1} \otimes V_{j_2} = V_{|j_1 - j_2|} \oplus V_{|j_1 - j_2| + 1} \oplus \cdots \oplus V_{j_1 + j_2}
\]

\end{enumerate}

\chapter{TD2}
\section{Properties of time-like vectors.}
\begin{enumerate}
\item Let $\vb{A}$ and $\vb{B}$ in $\mathcal{C}_+$. Then $a^0 > || \vec{a} || $ and similarly for $\vb{B}$. Hence $\vec{a}\cdot\vec{b} \leq || \vec{a}|| \cdot || \vec{b}|| \leq a^0 b^0$. Then $\vb{A} \cdot \vb{B} < 0$. 

\item Let $\vb{A}, \vb{B} \in \mathcal{C}_+$ and $\mu, \nu \in \mathbb{R}^+$ then $(\mu\vb{A} + \nu\vb{B})^2 = \mu^2\vb{A}^2 + 2 \mu \nu \vb{A} \cdot \vb{B} + \nu^2 \vb{B}^2 < 0$. Hence $(\vb{A} + \vb{B}) \in \mathcal{C}_+$.

\item A special Lorentz transformation is an isometry of the Minkowski space hence $\mathcal{C}_+$ is stable under it.

\item We have that:
\[
a^{i} - \beta^i a^0 = 0 \Rightarrow \beta^i = \frac{a^{i}}{a^0}
\]

\item Suppose by induction that this is true for $n$ the base cases being trivial. Then for $n+1$ note that $\mathcal{C}_+$ is stable under addition so any case can be reduced to the base case $n = 2$. We prove this case here:
\[
\sqrt{-(\vb{A} + \vb{B})^2} = \sqrt{-(\vb{A'} + \vb{B'})^2} = \sqrt{{d^0}^2} = d^0 
\]
Then $\vb{A_i}^2 = \vec{a_i}^2 - (a^0_i)^2$ and hence $a^0_i = \sqrt{- \vb{A_i}^2 + \vec{a_i}^2} \geq \sqrt{-\vb{A_i}^2}$ hence:
\[
\sqrt{-(\vb{A} + \vb{B})^2} \geq \sqrt{-\vb{A}^2} + \sqrt{-\vb{B}^2}
\]
\end{enumerate}

\section{Applications to 4-momenta}
\begin{enumerate}

\item  $\vb{P} = m \dv{\vb{X}}{\vb{\tau}} = (E, m \vec{U})$ and:
\[
\vb{P}^2 = - E^2 + m^2 \vec{U}^2 = -m^2 
\]

\item We directly have that $P^0 = E > 0$ and $\vb{P}^2 = -m^2 < 0$. Hence $\vb{P} \in \mathcal{C}_+$. 

\item From question 2 of Exercise 1 we know that since $\vb{P}_i$ are in $\mathcal{C}_+$ then so is $\vb{P}$. Then from question 4 of Exercise 1 we know that there exists a boost transformation such that $\vb{P} = (E^*, \vec{0})$. Then using question 5 of Exercise 1 we also know that:
\[
E^* \geq \sum_{i = 1}^n m_i 
\]

\end{enumerate}

\section{Decays of particles}

\begin{enumerate}

\item We must have that $M \geq \sum_{i = 1}^n m_i$. 

\item \begin{enumerate}

\item The number of unknowns are 8 since they are all the components of the two momenta $\vb{P}_1$ and $\vb{P}_2$. We also have the four equations given by: $\vb{P} = \vb{P}_1 + \vb{P}_2$. Finally we have two more equations $\vb{P_1}^2 = - m_1^2$ and $\vb{P_2}^2 = -m_2^2$. 

\item We have that:
\[
\vb{P_1}^2 = \vb{P}^2 + \vb{P_2}^2 - 2 \vb{P} \cdot \vb{P_2} \Leftrightarrow - m_1^2 = - M^2 - m_2^2 - 2 \left( - M E_2 \right) \Leftrightarrow 2 M E_2 = M^2 + m_2^2 - m_1^2  
\]
Then symmetry gives the desired opposite result.

\item We have:
\[
E_{kin, 1} = E_1 - m_1
\]
Which immediately gives the desired result after factorization and identically for $E_{kin, 2}$. Then:
\[
E_{kin, 1} + E_{kin, 2} = \Delta M 
\]
In other words all excess mass is converted to kinetic energy.

\end{enumerate}

\item For each new particle we get 4 more unknowns and one more equation so 3 more indeterminates. Now following the hint we write:
\[
\vb{P} = \sum_j \vb{P_j} = \vb{P_i} + \vb{Q}
\]
Then:
\[
\vb{P_i}^2 = \vb{P}^2 + \vb{Q}^2 - 2 \vb{P} \cdot \vb{Q}  \Leftrightarrow - m_i^2 = - M^2 - 2 M E' + \vb{Q}^2 
\]
Then we have:
\[
E_i = \frac{M^2 + m_i^2 + \vb{Q}^2}{2m} \mbox{~~and~~} E_{kin, i} = \frac{M^2 + m_i^2 - 2 M m_i + \vb{Q}^2}{2m}
\]
Now using question 5 of Exercise 1 we can bound $\vb{Q}^2$ as follows:
\[
\sqrt{-\vb{Q}^2} \geq \sum_{j \neq i} m_j \Rightarrow \vb{Q}^2 \leq -(M - \Delta M - m_i)^2
\]
Then re-injecting this above we get the desired inequalities. 


\end{enumerate}

\section{Creations of particles}

\begin{enumerate}

\item 

\end{enumerate}


\chapter{TD3}

\section{The Laplace Equation}

\begin{enumerate}

\item The solution is given by $\frac{q \vb{r}}{4\pi}$. 

\item Rotationally invariant harmonic functions are given by:
\begin{align*}
\laplacian u = 0 &\Leftrightarrow \dv{}{r} \left(r^{n-1} u'(r)\right) = 0 \Leftrightarrow r^{n-1} u'(r) = c \Leftrightarrow u'(r) = c\, r^{1 - n} \Leftrightarrow u(r) = \frac{c}{r^{n-2} (n - 2)} + c'
\end{align*}
When $n\neq 2$ in the case where $n = 2$ then we get:
\[
u(r) = c \ln r + c'
\]

\item We have that:
\begin{align*}
\int_\Omega \dd \vb{x} [u \laplacian v - v \laplacian u] &= \int_{\Omega} \dd \vb{x} \div [ u \grad v - v \grad u ] = \int_{\partial \Omega} \dd \vb{x}\, \vb{n} \cdot [u \grad v - v \grad u] = \int_{\partial \Omega} \dd \vb{x} \left[ u \pdv{v}{n} - v \pdv{u}{n} \right]
\end{align*}

\item We have that:
\begin{align*}
\int_{\overline{\mathcal{B}_\varepsilon}} \dd \vb{x}\, G(\vb{x}) \laplacian \varphi(\vb{x}) &= \int_{\overline{\mathcal{B}_\varepsilon}} \dd\vb{x} \left[ G(\vb{x}) \laplacian \varphi(\vb{x}) - \varphi(x) \laplacian G(x) \right]\\
&= \int_{\mathcal{C}_\varepsilon} \dd \vb{x} \left( G(\vb{x}) (-\vb{r}) \cdot \grad \varphi(\vb{x}) - \varphi(\vb{x}) (-\vb{r}) \grad G(\vb{x}) \right)\\
&= \int_{\partial \Omega} \dd \vb{x} -\varphi(\vb{x})\pdv{G}{r} \stackrel{\varepsilon \to 0}{\longrightarrow} \varphi(\vb{0}) \omega_n \varepsilon^{n-1} \pdv{G}{r}\Big|_{r = \varepsilon} = \varphi(\vb{0})
\end{align*} 

\item We have:
\[
\bra{G}\ket{\laplacian \varphi} = \bra{\delta}\ket{\varphi} = (-1)^2 \bra{\laplacian G}\ket{\varphi} = \bra{\delta}\ket{\varphi}
\]

\end{enumerate}

\section{The Helmholtz Equation.}

\begin{enumerate}

\item We have that:
\begin{align*}
(\laplacian + k^2) G_{\pm}(\vb{x}) &= -\frac{1}{4 \pi}(\frac{1}{r^2} \pdv{}{r} \left( r^2 \pdv{}{r}  \right) + k^2) \frac{e^{\pm i k r}}{r} = \frac{1}{4 \pi} \left( \frac{1}{r^2}\pdv{i e^{ik r} (i + kr)}{r} + k^2 \frac{e^{ikr}}{r} \right)\\
&= -\frac{1}{4\pi}\left( - \frac{e^{i k r} k^2}{r} + k^2 \frac{e^{ikr}}{r}  \right)
\end{align*}
Hence for all $r \neq 0$ where the differential is easily well defined it cancels.

\item Following the same steps as in part 1 questions 3 and 4 we get that $(\laplacian + k^2)G_{\pm}(\vb{x}) = \delta(\vb{x})$. 

\item We can easily deduce that the Green function of $-D$ is given by $-G_{\pm}$. Then up to taking $k = im$ we get the desired result. 

\end{enumerate}

\section{Fourier transforms}

\begin{enumerate}

\item We have that:
\[
DG = \delta \Leftrightarrow (1 + a_i \vb{\grad^i} + \cdots + a_{i_1, \cdots, i_p} \vb{grad^{i_1, \cdots, i_p}}) G = \delta \Leftrightarrow (1 + a_j (i p_j) + \cdots + a_{j_1, \cdots, j_p} ( i p_{j_1, \cdots, j_p}) \tilde{G} = 1
\]

\item We have that:
\[
\bra{p (\text{pv} \frac{1}{p} + \alpha \delta(p))}\ket{\varphi} = \varphi + \alpha \cdot \vb{0} \cdot \varphi(\vb{0}) = \varphi = \bra{1}\ket{\varphi}
\]

\item Let $\tilde{G}$ be a solution of $(9)$ then notice that:
\[
\bra{P(\vb{p}) (\tilde{G}(\vb{p}) + \alpha \delta(\vb{p} - \vb{p_0}))}\ket{\varphi} = \bra{1}\ket{\varphi} + \alpha \bra{P(\vb{p}) \delta(\vb{p} - \vb{p_0})}\ket{\varphi} = \bra{1}\ket{\varphi}
\]
In terms of $G(\vb{x})$ it corresponds to adding a constant.

\item 
\begin{enumerate}

\item From definition we have that $C_0 = \braket{\text{pv}\frac{1}{z}}{f}$. 

\item From the residue theorem we have that $C_{\pm} = \braket{\text{pv}\frac{1}{z} \mp i \pi \delta }{f}$.

\end{enumerate}

\item True because modifying the integral in a set of measure 0 changes nothing to the value of the integral.

\end{enumerate}

\section{The wave equation.}

\begin{enumerate}

\item Let $D = \pdv[2]{}{t} - \laplacian$ then:
\[
\tilde{D} = -\omega^2 - \laplacian = - (\laplacian + \omega^2)
\]
Then the equation becomes:
\[
\tilde{D} \tilde{G}(\omega, \vb{x}) = 1 \cdot \delta(\vb{x})
\]
Hence $-\tilde{G}(\omega, \vb{x})$ is a Green function of the Helmoltz operator.

\item We now know that:
\[
\tilde{G}(\omega, \vb{x}) = \frac{1}{4\pi} \frac{e^{\pm i \omega |\vb{x}|}}{|\vb{x}|}
\]
Then doing the inverse Fourier transform we obtain:
\[
G(t, x) = \frac{\delta(|\vb{x}| \pm t)}{4 \pi |\vb{x}|}
\]

\end{enumerate}


\chapter{Representations of the Lorentz group}

\begin{enumerate}

\item We have that:
\[
J_a = \mathcal{J}_a^L + \mathcal{J}_a^R \mbox{~~and~~} N_a = -i(\mathcal{J}_a^L - \mathcal{J}_a^R)
\]

\item We have:
\[
[\mathcal{J}_a^L, \mathcal{J}_b^L] = i \varepsilon_{abc} \mathcal{J}_c^L \mbox{~~and~~} 
[\mathcal{J}_a^R, \mathcal{J}_b^R] = i \varepsilon_{abc} \mathcal{J}_c^R
\]
Then we have that:
\[
[\mathcal{J}_a^L, \mathcal{J}_b^R] = 0
\]
Hence we have that $L_+^\uparrow$ can be seen as the product of two independent $SU(2)$ groups.

\item The dimension of a $j_R$ representation of $SU(2)$ is $2j_r + 1$ hence the $(j_R, j_L)$ representation of $L_+^\uparrow$ has dimension $(2j_r + 1)(2j_L + 1)$. 

\item We have that:
\[
i \theta^a J_a + i \nu^a N_a = i \theta_a (J_a^L + J_a^R) + i \nu^a i (J_a^R - J_a^L)
\]
Hence we have that:
\[
\rho(\Lambda) = e^{i(\theta_a + i \nu_a) \hat{J}_a^R + i(\theta_a - i \nu_a) \hat{J}_a^L} =  e^{i(\theta_a + i \nu_a) \hat{J}_a^R}  e^{i(\theta_a - i \nu_a) \hat{J}_a^L} \mbox{~~since~~} [\hat{J}_a^R, \hat{J}_a^L] = 0
\]
Hence we have that:
\[
\rho(\lambda) \ket{j_R, m_R} \otimes \ket{j_L, m_L} = e^{i(\theta_a + i \nu_a) \hat{J}_a^R} \ket{j_R, m_R} \otimes  e^{i(\theta_a - i \nu_a) \hat{J}_a^L} \ket{j_L, m_L}
\]

\item Notice that:
\[
\rho(e^{i \nu^a N_a})^\star = \left( e^{\nu^a(\hat{J}_a^L - \hat{J}_a^R)} \right)^\star = e^{\nu^a(\hat{J}_a^L - \hat{J}_a^R)} = \rho(e^{i \nu^a N_a}) \neq \rho(e^{i \nu^a N_a})^{-1}
\]
The boost are characterized by a parameter $\phi \in ]-\infty, \infty[$ and hence $L_+^\uparrow$ is not compact or similarly $\beta = \frac{v}{c} \in ]-1, 1[$ is bounded but not closed and hence not compact. 

\item The subgroup of $L_+^\uparrow$ containing only the rotation is the one generated by $J_1, J_2, J_3$. Or equivalently it is generated by $J_a^L + J_a^R$ and therefore is spanned by the $(j_R, j_L)$ representation. We have:
\[
\rho(e^{i \theta_a J_a}) = e^{i \theta_a (\hat{J}_a^R + \hat{J}_a^L)}
\]
Hence now defining $\hat{J}_a = \hat{J}_a^R + \hat{J}_a^L$ we have the sum of two angular momenta which we know decomposes $V_{j_R} \otimes V_{j_L}$ to $V_{|j_R - j_L|} \oplus V_{|j_R - j_L| + 1} \otimes \cdots \otimes V_{j_R + j_L}$. 

\item The dimension of the $(\frac{1}{2}, \frac{1}{2})$ representation is 4. From the rotation point of view this can be written as $0 \oplus 1$. This looks a lot like the $A^\mu$ representation which is given by: $\rho(\Lambda) A = \Lambda A$. 

\item We have that:
\[
(V_{j_R^1} \otimes V_{j_L^1}) \otimes (V_{j_R^2} \otimes V_{j_L^2}) = \left(\bigoplus_{i = |j_R^1 - j_L^1|}^{j_R^1 + j_L^1} V_i \right) \otimes \left(\bigoplus_{i = |j_R^2 - j_L^2|}^{j_R^2 + j_L^2} V_i \right) = \bigoplus_{i, j = |j_R - j_L|}^{j_R + j_L} V_i \otimes V_j
\]
In the special case of $j_R = j_L = \frac{1}{2}$ we obtain:
\[
\bigoplus_{i, j = 0}^1 V_i \otimes V_j = (0, 0) \oplus (0, 1) \oplus (1, 0) \oplus (1, 1)
\]

\item We have that:
\[
P J^{\gamma \sigma} 
\]

\end{enumerate}

\chapter{The Klein-Gordon equation}

\section{Basic properties.}



\end{document}