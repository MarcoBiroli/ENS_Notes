\documentclass[10pt,a4paper]{book}
\usepackage[utf8]{inputenc}
\usepackage[english]{babel}
\usepackage{amsmath}
\usepackage{amsfonts}
\usepackage{amssymb}
\usepackage{graphicx}
\usepackage[left=2cm,right=2cm,top=2cm,bottom=2cm]{geometry}
\usepackage{physics}
\usepackage{tikz}
\usepackage{stmaryrd}

\title{Symmetries in Physics}
\author{Marco Biroli}


\begin{document}
\maketitle

\chapter{TD1}
\section{Problem 1 Cayley tables}
\subsection{}
Suppose that an element appears more than once in a given row or column. Then we have that:
\[
\exists g, g_i, g_j, g_k \in \mathcal{G}, \quad g = g_i \cdot g_j \land g = g_i \cdot g_k \Rightarrow g_j = g_k
\]
Since no two elements in a row can be mapped to the same element of the group then a row is a map $\mathcal{G} \to \mathcal{G}$ which from the point above is injective then since it is an endomorphism it necessarily must be a bijection and hence a permutation of $\mathcal{G}$. Therefore each element appears once and exactly once.

\subsection{}
Refer above.

\section{Problem 2 The group $D_3$}
\subsection{}
The elements of $D_3$ are $e = Id$, $r = (B, C, A)$, $r^2 = (C, A, B)$, $s_1 = (A, C, B)$, $s_2 = (B, A, C)$, $s_3 = (C, B, A)$. Then the table is given by:
\begin{center}
\begin{tabular}{ c | c | c | c | c | c | c }
 & $e$ & $r$ & $r^2$ & $s_1$ & $s_2$ & $s_3$\\
 \hline
 $e$ & $e$ & $r$ & $r^2$ & $s_1$ & $s_2$ & $s_3$ \\
 \hline
 $r$ & $r$ & $r^2$ & $e$ & $s_2$ & $s_3$ & $s_1$ \\
 \hline 
 $r^2$ & $r^2$ & $e$ & $r$ & $s_3$ & $s_1$ & $s_2$\\
 \hline
 $s_1$ & $s_1$ & $s_2$ & $s_3$ & $e$ & $r$ & $r^2$\\
 \hline
 $s_2$ & $s_2$ & $s_3$ & $s_1$ & $r^2$ & $e$ & $r$\\
 \hline 
 $s_3$ & $s_3$ & $s_1$ & $s_2$ & $r$ & $e$ & $r^2$
\end{tabular}
\end{center}

\subsection{}
The subgroups of $D_3$ are $\{e, r, r^2\} = \langle r \rangle$, $\langle s_1 \rangle$, $\langle s_2 \rangle$, $\langle s_3 \rangle$, $\{e\}$. The quotient groups are: ...

\section{Problem 3 Lagrange's theorem.}
Let $\mathcal{H}$ be a subgroup of $\mathcal{G}$. Then notice that $\mathcal{G} / \mathcal{H}$ is the set of the cosets of $\mathcal{G}$ by the congruence modulo $\mathcal{H}$. However from Exercise 1 we know that every coset is in bijection with $\mathcal{H}$. Furthermore since the congruence is an equivalence relation it must be that $\mathcal{G}$ is equal to the reunion of the cosets. Hence we have that:
\[
|\mathcal{G} / \mathcal{H} | \cdot |\mathcal{H} | = | \mathcal{G} | 
\]
The result follows.

\section{Problem 4 Modular arithmetics}
\subsection{}
Notice that for any $k \in \mathbb{Z}$ we have that $k\mathbb{Z}$ is a subgroup of $\mathbb{Z}$. The quotient groups are $\mathbb{Z}_{k\mathbb{Z}}$ which are the well-known integers modulo $k$ with the addition modulo $k$.

\subsection{}
Let $|g|$ be the smallest integer such that $g^{|g|} = e$. Such an integer must exist so long as the group to which $g$ pertains is finite. Then notice that for any $k \in \mathbb{Z}$ we have that: $g^{|g|\cdot k} = (g^{|g|})^{k} = e^k = e$. Hence $|g|\mathbb{Z} \subseteq P_g$. Now let $k \in \mathbb{Z}$ such that $g^k = e$. From construction it must be that $k > |g|$ hence by doing the euclidean division we get that : $k = |g|\cdot \ell + r$. Hence: $g^{|g| \cdot \ell + r} = e \Rightarrow e^\ell \cdot g^r = e \Rightarrow g^r = e$. However unless $r = 0$ this is impossible since $r < |g|$ would be a contradiction.

\subsection{}
Notice that necessarily $\langle g \rangle$ is a subgroup of cardinality $|g|$ of $\mathcal{G}$ hence from the Lagrange theorem we know that $|g|$ divides $|\mathcal{G}|$.

\subsection{}
Let a group $\mathcal{G}$ of order $p$ where $p$ is prime. Then from the previous question we know that all elements of $\mathcal{G}$ must be of order $p$. However if one element is of order $p$ and $\mathcal{G}$ is of order $p$ it must be that $\mathcal{G}$ is generated by a single element, call it $g$. Then the obvious homomorphism concludes the proof:
\begin{align*}
h : \mathcal{G} &\to \mathbb{Z}/p\mathbb{Z}\\
g^k &\mapsto k \mod p
\end{align*}



\end{document}